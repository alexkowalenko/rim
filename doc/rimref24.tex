% Rim reference manual in LaTeX [Writeup] format
% by Jim Fox, last updated: February 1990
%
%   Permission is granted to process this file through Tex and
%   print the results, provided the printed document carries this
%   permission notice and the notice defined in \UCScopytext below.
%
\documentclass[11pt,a4paper]{report}
%
\def\opt#1{$\langle \mbox{#1} \rangle$}
\def\Opt#1{$\Bigl\langle \mbox{#1} \Bigr\rangle$}
\def\stk#1{\begin{tabular}[c]{c}#1\end{tabular}}
%
\def\asisbox#1#2{\setbox#1=\vbox\bgroup\hsize#2}
\def\asistopbox#1#2{\setbox#1=\vtop\bgroup\hsize#2}
\def\endasisbox{\egroup}

\def\UCS{University Computing Services}
\def\I{\index}

\def\List{\list{}{\leftmargin2\leftmargini
      \def\makelabel##1{\bf##1\hfill}}}
\let\endList\endlist

\usepackage{a4wide}
\usepackage[colorlinks,allcolors=blue]{hyperref}
%
%
\raggedbottom
\predisplaypenalty 10000
\textfloatsep 30pt
%
%

\makeindex

\begin{document}
%
\author{Jim Fox}
%\UCScopy  {1988, 1989, 1990}
\title {RIM Users Manual}
%\UCSdoc   {A700}
\date   {February 9, 1990}

\def\UCScopytext{%
  Permission is granted to make and distribute verbatim copies of
  this manual provided the copyright notice and this permission notice
  are preserved on all copies.
  \par
  Permission is granted to copy and distribute modified versions of this
  manual under the conditions for verbatim copying, provided also that the
  sections entitled ``Distribution'' and
  ``Rim General Public License'' are included exactly as in the original, and
  provided that the entire resulting derived work is distributed under the
  terms of a permission notice identical to this one.
  \par}
\maketitle
%
%
\tableofcontents
\listoffigures

\abstract{This document tells you how to use Rim---University
  Computing Services' relation database management system,
  available on all of the mainframe computers at the UCS.
  You are assumed to be able to access the computer of your choice,
  and be able to edit text files.
  Chapter~\ref{pi-chapter}, which discusses the program interface, assumes
  you are familiar with \textsc{Fortran-77}.}

%
% no warranty
%
\chapter*{No Warranty}
  \markboth{No Warranty}{No Warranty}
  \addcontentsline{toc}{chapter}{No Warranty}
%
{\parindent0pt\parskip1pc
  Because Rim is licensed free of charge, we provide absolutely
  no warranty, to the extent permitted by applicable state law.
  The University of Washington, University Computing Services,
  or other parties provide Rim ``as is''
  without warranty of any kind, either expressed or implied, including,
  but not limited to, the implied warranties of merchantability and
  fitness for a particular purpose.  The entire risk as to the quality
  and performance of the program is with you.  Should Rim
  prove defective, you assume the cost of all necessary
  servicing, repair or correction.
 
  In no event unless required by applicable law will
  the University of Washington, University Computing Services,
  or any other party who may
  modify and redistribute Rim as permitted above, be liable to you
  for damages, including any lost profits, lost monies, or other
  special, incidental or consequential damages arising out of the use or
  inability to use (including but not limited to loss of data or data
  being rendered inaccurate or losses sustained by third parties)
  the program, even if you have been
  advised of the possibility of such damages, or for any claim by any
  other party.
}
%
% beginning of the chapters
%

%
%  <<<<< Introduction >>>>>>>>>>
%
\chapter{Introduction}
 
Rim is a popular and easy-to-use
relational database management package available
on all of the \UCS\ mainframe computer systems.
It was specifically designed to be easily transported between
computers and operating systems without the need for
customization.  Rim users will find the same command language,
functionality, and reliability in all implementations.
 
 
\section{Rim is a relational database manager}
\I{relational database}
A relational database may be thought of as a collection of
one or more tables.  These tables (also called <relations> in
database terminology) consist of rows (\emph{tuples}) and
columns (\emph{attributes}).
This book will use the terms table,
row, and column.\footnote{The term `column' is also used for
the columns of a matrix.  This may cause hesitation, but
should not cause confusion.}
Rim allows the values in a table to be
scaler, vector, and matrix integer or real numbers;
character strings; dates; or times.
The vectors, matrices, and character strings may be of fixed
or variable length.
 
A table is defined with a fixed number and sequence
of columns.  The \verb|conductors| table in
figure \ref{Sample-conductor}, for example,
is defined with four columns: \verb|Atomic_No.|, an integer;
\verb|symbol|, a fixed length character string;
\verb|resistivity|, a double precision real number;
and \verb|name|, a variable length character string.
As data are added to, deleted from, or changed in the
table the number and sequence of rows will vary, but
the number and sequence of the columns will remain the same.
The "conductors" table contains five rows.
 
\begin{figure}[htp]
\begin{center}
  \begin{tabular}[t]{|r|l|r|l|}
     \hline
     \multicolumn{1}{|c|}{atomic\_no.}&
     \multicolumn{1}{|c|}{symbol}&
     \multicolumn{1}{|c|}{resistivity}&
     \multicolumn{1}{|c|}{name}\\
     \hline
    29& Cu& 12.2  & Copper\\
    13& Al& 12.9  & Aluminum\\
    26& Fe& 20.3  & Iron\\
--NA--& LB&--NA--& Bernstein\\
    92& U &--MV--  & Uranium\\
     \hline
  \end{tabular}
    \par  {<conductors> table }
  \end{center}
\smallskip
\caption[{\tt conductors} table of the Sample database]
    {The {\tt conductors} table of the Sample database.
    ``-NA-'' indicates that the column is not applicable for
    the row.
    ``-MV-'' indicates that a value for the column is missing.
    The complete Sample database is shown in figure~\ref{Sample}
    on page~\pageref{Sample}.}
   \label{Sample-conductor}
\end{figure}
 
 
The primary advantage of the relational database is its
structural simplicity.  There is no need for the user to
learn the parent-child relationships found in hierarchical
databases, nor is there need to understand the pointers
of a network database.
Data are always represented by simple tables.
 
 
 
 
 
 
 
\section{SQL---a database language}
\I{SQL}
The Structured Query Language (SQL), pronounced `<see-quel\/>',
is a command syntax for relational
database access.  It uses keywords and an ``English'' syntax
to construct database commands.  For example,
 
\begin{verbatim}
     select symbol, resistivity from conductors where
         resistivity lt 14;
\end{verbatim}
 
\noindent
is a SQL command which selects those rows from the ``conductors''
table that have a resistivity less than 14.
SQL has been proposed by the American National Standards Institute (ANSI)
\I{ANSI}
as the basis for a standard relational database language.
 
 
Rim does not attempt to implement a precise or complete SQL.
Instead, it uses SQL as a guide, so its command language
may be best described as ``SQL-like''.
Although you can tailor Rim to accept the particular
SQL command above,\footnote{The `comment' {\tt  *(set continue=null)}
 \I{comments}
 accomplishes this feat. See section \ref{cmt-cmd}
 for details on the `comment' commands.}
 the analogous, normal Rim command is
 
\begin{verbatim}
     select symbol  resistivity from conductors where
         resistivity lt 14
\end{verbatim}
 
\noindent
It has few of minor differences from the SQL syntax:
a plus sign
indicates line continuation and no explicit command terminator is required.
Chapter \ref{cmd-chapter} describes the Rim command language
in detail.
 
\I{SQL}
The principal advantage to Rim users of a language like SQL
(versus screens or menus) is the universality
and commonality of the dialog interface.  Rim's appearance
is the same, regardless of the computer, terminal, or operating
system.
A second advantage is that users already familiar with
SQL will not have to learn a new language to use Rim.
 
\section{Relational Algebra}
\I{algebra} \I{relational database}
An algebra is defined for relational databases which will
create new tables from old.  Here is a brief look
at the operations of this algebra:
 
\begin{List}
\item[union:] The resultant table contains all rows from
  two source tables.
 
\item[intersection:] The resultant table contains only  those rows from
  two source tables where like columns have equal values.
 
\item[subtraction:] The resultant table contains only those rows from
  one source table which are not also in the other source table.
 
\item[join:] The resultant table's rows contain columns from
  two source tables.  The rows are matched by comparing a specified
  column in each table.
 
\item[projection:] The resultant table is a subset of a single
  source table.  It may have fewer columns and fewer rows
  than the original table.
 
\end{List}
 
These operations are discussed in detail in section
\ref{ra-section}, which describes the Rim relational
algebra commands.
 
 
\section{Program Language Interface}
\I{Fortran}
Rim provides a convenient function interface
for users who need to access a Rim database from
within their programs.  The interface provides one \textsc{Fortran-77} function
the program to issue Rim commands and another
function transfers data between Rim and the user's program.
Chapter \ref{pi-chapter} describes the Rim program interface
in detail.
 
\section{Should you use Rim?}
Rim provides a convenient program interface,
relational data access,
and much support for loading, unloading,
maintaining, and reporting on databases.  
It allows your application to be very
portable---as long as you keep it clean of system dependencies.
It does not, however, present
a very convenient user interface for adding or changing data.
And it has no screens.  

Rim is most useful for databases that are not updated frequently,
OR for dynamic databases that will be updated by user programs.

\section{A Brief History of Rim}
Rim is a descendent of the Boeing Computer Services\I{Boeing} program
of the same name that was developed in 1978 as part of the IPAD project
(NASA contract NAS1-14700)\I{NASA}.
That program was brought
to the University of Washington and further developed
as UWRIM\I{UWRIM}, a CDC Cyber program.
The present program has been rewritten in completely portable
\textsc{Fortran-77} code.  The new code includes new functionality
and a more ``SQL-like'' command language.
Some capabilities of UWRIM are no longer supported.
 
%
% <<<<<< Database >>>>>>>>>>>>
%
\chapter{What is a Rim database?}
 
 
This chapter describes a Rim database in detail.
 
 
The Rim database consists of all the column and table
definitions, all of the table data (rows),
and supplementary information such as passwords.
 
\I{Rim name}
Many elements of a Rim database are identified by a name:
\begin{itemize}
  \item  the database name,
  \item  owner and table passwords,
  \item  table names, and
  \item column names.
\end{itemize}
These names consist of one to sixteen
characters (letters, digits, and the underscore are allowed).
They may contain both upper and lower case letters but Rim
will always disregard the case of letters when comparing the names.
`\verb|conductors|' and `\verb|Conductors|' are valid names.  They are also
identical.  `\verb!my text!' is not a valid name because
a space is not allowed.
 
\section{Columns}
\I{column}
A column in a Rim database identifies both a particular
column of a table and the type of data contained in that
column.  The column \verb!symbol! in the Sample database
(figure~\ref{Sample} on page~\pageref{Sample}),
for example, identifies particular
columns in the ``conductors'', ``measures'', and ``notes'' tables.
It also represents a data type of text with a length of
eight characters.
 
These are the data types supported by Rim.
 
\begin{List}
\item[text]\I{text} A text column is a fixed or variable length
   character string.  Rim uses the ASCII character representation
   on all machines.
\item[integer]\I{integer} An integer column is a one-word integer.
\item[integer vector]\I{vector} An integer vector column is
   a fixed or variable length, one-dimensional integer array.
\item[integer matrix]\I{matrix} An integer matrix column is
   a two-dimensional integer array.  It may have either
   fixed length rows and columns, fixed length rows and variable
   length columns, or variable length rows and columns.
\item[real]\I{real} A real column is single
   floating point number.
\item[real vector]\I{vector} A real vector column is
   a fixed or variable length, one-di\-men\-sional
   array of floating point numbers.
\item[real matrix]\I{matrix} A real matrix column is
   a two-dimensional array of floating point numbers.
   It may have either
   fixed length rows and columns, fixed length rows and variable
   length columns, or variable length rows and columns.
\item[double]\I{double precision} A double column is single,
   double-precision, floating point number.
\item[double vector]\I{vector} A double vector column is
   a fixed or variable length, one-dimensional array of
   double-precision, floating point numbers.
\item[double matrix]\I{matrix} A double matrix column is
   a two-dimensional, double-precision array.  It may have either
   fixed length rows and columns, fixed length rows and variable
   length columns, or variable length rows and columns.
\item[date]\I{date} A date column is a Julian integer value.
   It is input and reported in a user specified format
   (including year, month, and day).
%%%   FROM ACM-CA 199, BY R. TANTZEN
\item[time]\I{time} A time column is a integer value containing
   the number of seconds from midnight.
   It is input and reported in a user specified format
   (including hour, minute, and optional second).
\end{List}
 
Each column has an optional default format, which will be used
by Rim during formatted input or output if no explicit
format is otherwise specified.
 
\section{Tables}
\I{table}
A Rim table is defined as a sequence of columns.
Rows may be added to or deleted from the table, but the
order and number of columns is invariant.
 
Each row of the table contains one data item for each
column.\footnote{Note that one data item for a vector or
matrix column may consist of several numbers, and one data item
for a text column may consist of many characters.}
The data may be an actual value or it may be a `missing value' code.
The "conductors" table in Figure \ref{Sample} is an example
of a table with four columns and five rows.
 
 
 
\section{Links}
\I{link}
A link is a logical connection between
each row of a source table and a unique
row of a destination table.
It is most commonly used to associate an identifier,
\verb!tech_id! in the \verb!measurements! table of
figure~\ref{Sample}, for example,
with information about that identifier---a row in the
\verb!technicians! table in this case.
 
\section{Passwords}
\I{password}
Access to tables is optionally protected by passwords.%
\footnote{Users may also be restricted by permissions granted,
or not granted,
by the operating system.}
Rim provides three levels of password protection:
 
\begin{List}
\item[owner password,] specified by the \verb!define owner! sub-command,
restricts access to database modification commands and to protected
tables.
Users who have not entered the owner password (\verb!user! command)
are unable to modify the database and are required to
enter the appropriate table password for access to
protected tables.
Users who have correctly entered the owner password (\verb!user! command)
have unrestricted access to the database.
\item[table read password,]
specified by the \verb!define passwords! sub-command,
prevents unauthorized users from reading the protected table.
Permission to read is granted only to those users who have entered
either the owner password or the table's read password.
\item[table modify password,] also
specified by the \verb!define passwords! sub-com\-mand,
prevents unauthorized users from modifying the protected table.
Permission to modify is granted only to those users who have entered
either the owner password or the table's modify password.
\end{List}
 
See section \ref{ocmd-section} and \ref{ucmd-section} for
complete descriptions of the \verb!owner! and \verb!user! commands.
 
\section{Keys}
\I{key}
A key is an ordered list of pointers to the rows of
a table---much like the index of a book.
Existence of a key for an column can greatly
facilitate Rim's access to data rows during retrievals.
However, because keys need to be maintained, their
existence is detrimental to Rim's efficiency during database update.
 
Normally you will build a key for those columns
of a table which are most often referenced in query commands.
Rim will, in some circumstances, automatically build a key.
Columns for which keys have been built are referred to as
``keyed columns''.
 
 
 
\section{Missing Values}
\I{missing value}\I{not applicable value}
Occasionally a partial row must be added to a table.
This partial row will have data for some columns but
will not have data for others.  Rim provides two distinct
missing values, distinguished by their codes:
`\verb!-MV-!', for `missing value',
and `\verb!-NA-!' for ``not applicable'' value.
The \verb!conductors! table of the Sample database (page~\pageref{Sample})
contains one missing value and two values that are not applicable.
 
\section{Where is it?}
The database is contained on three direct access files
on the user's disk area.  The actual naming of these files
is dependent upon the capabilities and conventions of
the particular operating system.
\I{filename}
Generally the
files have a common name, which is the database identifier,
and extensions of {\tt rimdb1}, {\tt rimdb2}, and
{\tt rimdb3}.
The content of these files is the responsibility of Rim,
but the user is responsible for any copies or backups
of the database.
The files are not in text format and cannot be directly
edited.
 
 
 
 
%  <<<<<< Commands >>>>>>>>>>

\chapter{Using Rim}
\label{cmd-chapter}
This chapter describes the command language of Rim.
As discussed in the introduction, this language is similar to
SQL but is not an exact implementation.\I{SQL}
To see how to run Rim on your computer, consult Appendix~\ref{UCS-systems}.
 
\I{Rim command}
A Rim command consists of a command name that is sometimes followed by
keywords and qualifiers (table names, column names, data values,
filenames, etc.).
Input is free format with one or more spaces delimiting
the items.
A trailing plus sign (\verb!+!) continues a command on
the next line.
 
In addition to the space and plus, these characters have
special meaning to Rim.

{\tt (~)~[~]~,~;~:~\char`\<~\char`~@~\%~= 
\rm and sometimes \tt ' \char`\"}

You must enclose these characters in quotes 
(either {\tt\char`\'}---{\tt\char`\'} or
{\tt\char`\"}---{\tt\char`\"}) to enter them as text.

\medskip
 
This book uses the following conventions to
describe the syntax of Rim commands.
 
\medskip
\begin{quote}
    {\centering \fbox{ |Bold text|} \par}
  \nopagebreak \medskip
  indicates that you must type the command or keyword
  exactly as shown.  Except that you may abbreviate keywords
  to three characters. For example,
  \verb!sel! is a suitable abbreviation for \verb!select!.
 
  \bigskip
 
     {\centering \fbox{\emph{Italic text}}\par}
  \nopagebreak \medskip
  indicates that you must enter the name of something
  For example, if \emph{table} is specified,
  you must supply the name of a specific table.
 
  \bigskip
 
     {\centering  \fbox{\opt{<Angle brackets>}}\par}
  \nopagebreak \medskip
  denote an optional word or phrase.
 
  \bigskip
 
     {\centering \fbox{\stk{Stacked\\items}}\par}
  \nopagebreak \medskip
  indicate that you must enter one
  of the items in the stack.
 
  \bigskip
 
{\centering \fbox{\strut\ \ldots\ }\par}
  \nopagebreak \medskip
  indicate that you may enter more than one of the previous item.
  \bigskip
 
     {\centering \fbox{\strut\ \vdots\ }\par}
  \nopagebreak \medskip
  indicate that you may enter more than one of the previous line.
 
\end{quote}
 
In addition, {\tt typewriter text} is used to indicate specific
examples of commands.
 
Many of these conventions may be used in a single command description.
This example

"select" \stk{<column>\opt{"["<col>\opt{","<row>}"]"}%
          \opt{"@"<fmt>} \ldots \\
             "*"}
     "from" <table>

indicates that these are valid commands.
 
\begin{verbatim}
select name from conductor
select index@i2 name from conductor
select * from conductor
\end{verbatim}
 
Of course the \verb![col]! and
\verb![<col>,<row>]! are valid only for vector and matrix
columns, respectively.
 
 
\section{Obtaining help}
\I{help@"help"}
Rim supplies online help text which describes the
syntax and functionality of commands,
 typing conventions, and also provides general information about Rim.

Enter
\begin{verse}
\verb!help!
\end{verse}

to view an introduction to Rim and a list of Rim commands.

Enter
\begin{verse}
\verb!help typing!
\end{verse}
to see help text about Rim's command and data entry conventions.

Enter
\begin{verse}
\verb!help! \opt{\emph{command} \opt{\emph{sub-command}}}
\end{verse}
to see help text related to a specific Rim command.
 
\section{Including comments in Rim commands}
%
\I{comments}
You may include a comment nearly anywhere in your Rim input
by enclosing it in `{\tt*(}' and `{\tt )}'.
end{verse}
\begin{verse}
part of command  "*({\em This is a comment})"  rest of command
\end{verse}

You may not put comments in quoted text strings or in formatted
input data.
The Rim commands which load the Sample database
(figure~\ref{Sample-loadf} on page~\pageref{Sample-loadf})
contain comments.
 
There is also a comment command.
\verb!*! <rest of command>
\I{"*" (command)@"*"\ (command)}
The <rest of command> is parsed, so normal input rules must
be followed, but no action is taken.  This command is 
used for comments, but is also useful with the "echo" mode
setting (page~\pageref{echo-mode}) to show macro expansions
\I{macros}
(see chapter~\ref{mac-chapter}).
 
\section{Defining a database}
\label{ocmd-section}
This section tells you how to create, change, link, and
remove tables.
You must have entered the owner password
(if the database has one) before you can use
database definition commands.
 
\subsection{Creating tables}
Create new tables, or change the definition of
existing tables, with a block of commands that begins with
``define'' and ends with ``end''.  The block is show
in figure \ref{define}.
 
\begin{figure}[htp]
  \verb!define! <filename> \opt{file parameters}\\
   \opt{name <name>}\\
   \opt{owner <password>}\\
   \Opt{\begin{tabular}{l}columns\\
               <column definitions>\end{tabular}}\\
   \Opt{\begin{tabular}{l}tables\\
               <table definitions>\end{tabular}}\\
   \Opt{\begin{tabular}{l}links\\
               <link definitions>\end{tabular}}\\
   \Opt{\begin{tabular}{l}passwords\\
               <password definitions>\end{tabular}}\\
   \verb!end!
\caption[Table definition block]
 {The block of commands that begin with {\tt define} and end with
 {\tt end} define or modify the definition of a database.}
\label{define}
\end{figure}
 
\subsubsection{Beginning the definition}
%
\verb!define! \opt{<filename> \opt{<parameters>}}
\I{define@"define"}
begins definition of a new database if the files do not exist, or
begins modification of an existing database if the files do exist.
Some systems allow a parameter following the filename to further
identify the files.\footnote{For instance, VM/CMS users can
specify the files' filemode after the name.}
Consult Appendix~\ref{UCS-systems}
to see if your system allows this parameter.
 
If you are defining a new database, the files are created
and the database name becomes the filename.
 
If a database is currently open, the filename may be omitted
and the current database will be edited.
 
\subsubsection{Naming the database}
%
Enter the define command
\verb!name! <name>
\I{define!name@"name"}
\I{name@"name"}
to give the database a new name.  This does not affect the
files on which the database resides.
 
 
\subsubsection{Specifying an owner password}
%
Enter the define command
\verb!owner! <password>
\I{define!owner@"owner"}
\I{owner@"owner"}
to identify the owner password of the database.
Use this command only if
 
\begin{enumerate}
\item your user password is not the same as the owner password
   of an existing database, or
\item you want to add an owner password to a new or existing
   database.
\end{enumerate}
 
\subsubsection{Entering columns definitions}
%
Enter the define command
\verb!columns!
\I{define!columns@"columns"}
\I{columns@"columns"}
to begin entry of column definitions.
All entries until the next \verb!define! subcommand
are column definitions.

There are four styles of column definition lines.
They have the general syntax
\begin{verse}
<name> <type> \opt{<length>} \opt{format <format>}
\end{verse}
where <format> is the optional, default format for the column.
Table \ref{formats} shows the syntax of format specifications.
\I{format}
 
\begin{figure}[t]
\centering
\begin{tabular}{|c|c|p{18pc}|}
   \hline
     Data type&
     Syntax&
     Definitions\\
     \hline
     <text> & \verb!A!<xx>&
            <xx> = field width for formatted input and output.
            Output text longer than <xx> characters will be continued
            on subsequent lines.  Input text longer than <xx>
            characters will be truncated.\\
     \stk{<int> \\ <real>} &
       \stk{\opt{<r>}I<n>\opt{"."<d>} \\
          \opt{<r>}F<n>\opt{"."<d>} }&
          \hangindent24pt\hangafter1
       <r> = repeat count for vector and matrix output.
            Output longer than <r> numbers will be paragraphed.
            Input longer than <r> numbers will be truncated.\par
       <n> = field width per number.\par
          \hangindent24pt\hangafter1
       <d> = decimal places.
            If <D> is specified for integer input, the value
               will be multiplied by $10^d$ after input.
            If <D> is specified for integer output, the value
               will be divided by $10^d$ before output.\par\\
     <date> & <string>& <string> = a reasonable combination of
                     "dd", "mm" or "mmm", and "yy" or "yyyy".
                     "mm/dd/yy" and "mmm-dd-yyyy" are valid
                     date formats. \\
     <time> & <string>& <string> = a reasonable combination of
                     "ss" (optional), "mm", and "hh".
                     "hh:mm:ss" and "hh:mm" are valid
                     time formats. \\
   \hline
\end{tabular}
\caption{Column format syntax}
\label{formats}
\end{figure}
 
 
Define a text column with
\begin{verse}
\emph{name}
             \verb!text!
             \stk{<\#chars> \\ var}
             \opt{format <format>}
\end{verse}
where \verb!var! indicates a variable length string.
Table \ref{formats} shows the syntax of format specifications.
 
Define a scalar column with
\begin{verse}
\emph{name}
             \stk{int\\real\\double\\date\\time}
             \opt{format <format>}
\end{verse}

Define a vector column with
\begin{verse}  
\emph{name}
             \stk{ivec\\rvec\\dvec}
             \stk{<\#cols> \\ var}
             \opt{format <format>}
\end{verse}
where \verb!var! indicates a variable length vector.
 
Define a matrix column with
\begin{verse}
\emph{name}
             \stk{imat\\rmat\\dmat}
             \stk{<\#rows>, <\#cols> \\ <\#rows>, var \\ var, var}
             \opt{format <format>}
            \end{verse}            
where \verb!var! indicates a variable length matrix row or column.
 
 
\subsubsection{Entering table definitions}
%
Enter the define command
\begin{verse}
\verb|tables|
\I{define!tables@"tables"}
\I{tables@"tables"}
\end{verse}
to begin entry of table definitions.
All entries until the next \verb!define! subcommand
are table definitions.
Each entry contains a table name followed by a list of columns
\begin{verse}
<name> \verb!with! <column> \opt{<column> \opt{\ldots}}
\end{verse}
where each <column> has been defined in the
\verb!columns! section or already exists in the database.
 
\subsubsection{Specifying links between tables}
%
Enter the define command
\begin{verse}
\verb|links|
\I{define!links@"links"}
\I{links@"links"}
\end{verse}
to begin entry of link definitions.
All entries until the next \verb!define! subcommand
are link definitions.
Each has the following syntax.
\begin{verse}
<link name> \verb!from! <att1> \verb!in! <rel1> \verb!to! <att2> \verb!in! <rel2>
\end{verse}
where the value of column <att1> in table <rel1>
will be used to find a unique row of table <rel2>,
where <att2> = <att1>.
 
 
Links are used by the \verb!select! command.
 
 
\subsubsection{Ending database definitions}
%
Enter the define command
\begin{verse}
\verb|end| \opt{define}
\I{define!end@"end"}
\I{end@"end"}
\end{verse}
to end database definition and returns to normal Rim command
processing with the newly defined database open.
 
Figure~\ref{Sample-def1}  on page~\pageref{Sample-def1} shows the input that
defined the tables in the sample database.
 
 
\subsection{Changing names}
\I{rename@"rename"}
You can change the name of a table or column with the "rename" command.
To rename a table
enter
\verb|rename| table <table> to <new\_name>"
To rename a link
enter
\verb|rename| link <link> to <new\_name>"
To rename a column
enter
\verb|rename| \opt{"column"} <column> "to" <new\_name> \opt{"in" <table>}
If a table name is given, the
column is renamed only in that table. Otherwise the column is
renamed in all tables.
 
\subsection{Changing a default format}
\I{reformat@"reformat"}
\I{format}
You can change the default format of a column with the "reformat" command.
"reformat" <column> "to" <format> \opt{"in" <table>}
changes the default format for the specified column.
If a table name if given, the
format is changed only in that table. Otherwise it
is changed in all tables.
 
\subsection{Removing tables and links}
%
You can remove a table or link from the database with the "remove" command.
\I{remove table@"remove table"}
\verb|remove| \opt{"table"} <table\_name>
removes the specified table.  All data rows contained in the
table will be lost.  Any links to or from the table
will also be removed.
If you are running Rim interactively, Rim will ask for
confirmation before removing a table.
 
Removes a link with the command
\I{remove link@"remove link"}
\verb|remove| link" <link\_name>
 
\subsection{Defining the "Sample"\ database}
\I{Sample}
Figure~\ref{Sample} on page~\pageref{Sample} shows an example
database that is used
with this reference manual.  Figure~\ref{Sample-def1}
shows the input that was used to create the Sample database.
Figure~\ref{Sample-def2}
shows the input that was used to modify the Sample database.
 
 
\begin{figure}[htp]
\begin{center}
  \begin{tabular}[t]{|r|l|r|l|}
     \hline
     \multicolumn{1}{|c|}{atomic\_no.}&
     \multicolumn{1}{|c|}{symbol}&
     \multicolumn{1}{|c|}{resistivity}&
     \multicolumn{1}{|c|}{name}\\
     \hline
    29& Cu& 12.2  & Copper\\
    13& Al& 12.9  & Aluminum\\
    26& Fe& 20.3  & Iron\\
--NA--& LB&--NA--& Bernstein\\
    92& U &--MV--  & Uranium\\
     \hline
  \end{tabular}
 
  \smallskip
          {<conductors> table }
 
  \bigskip
  \begin{tabular}[t]{|r|l|l|l|}
     \hline
     \multicolumn{1}{|c|}{id}&
     \multicolumn{1}{|c|}{name}&
     \multicolumn{1}{|c|}{position}\\
     \hline
     5& John Jones& Tech 1\\
    22& Jim Smith& Tech 2\\
    35& Joe Jackson& Tech 1\\
     \hline
  \end{tabular}
 
  \smallskip
          {<Techs> table }
 
  \bigskip
  \begin{tabular}[t]{|l|r|l|l|r|}
     \hline
     \multicolumn{1}{|c|}{symbol}&
     \multicolumn{1}{|c|}{id}&
     \multicolumn{1}{|c|}{date}&
     \multicolumn{1}{|c|}{time}&
     \multicolumn{1}{|c|}{resistivity}\\
     \hline
    Cu& 5& 88-01-21& 08:10& 11.9 \\
    Cu& 5& 88-01-21& 10:32& 12.4 \\
    Al&35& 88-02-10& 13:45& 13.0 \\
    Al&35& 88-02-11& 09:34& 14.3 \\
    Cu&22& 88-02-22& 08:48& 12.5 \\
    Fe& 5& 88-03-04& 15:33& 19.4 \\
     \hline
  \end{tabular}
 
  \medskip
          {<measures> table }
 
  \bigskip
  \begin{tabular}[t]{|l|r|l|l|p{2.0in}|}
     \hline
     \multicolumn{1}{|c|}{symbol}&
     \multicolumn{1}{|c|}{id}&
     \multicolumn{1}{|c|}{date}&
     \multicolumn{1}{|c|}{time}&
     \multicolumn{1}{|c|}{notes}\\
     \hline
    Cu& 5& 88-01-21& 09:03& Spilled coffee on sample.  Expect higher
        resistivity due to poor electrical contact.\\
    Cu&22& 88-02-22& 09:20& Sample seems contaminated with some
        sort of dusty deposit.\\
     \hline
  \end{tabular}
 
  \smallskip
          {<notes> table }
\end{center}
\caption[The Sample database]
  {The Sample database used by all the example commands
   in this reference manual.}
\label{Sample}
\end{figure}
 
 
 
\begin{figure}[htp]
\begin{verbatim}
define Sample
columns
  at_no         int             format i4
  symbol        text     8
  resistivity   double          format f7.6
  name          text     var    format a12
  M_date        date            format 'yy-mm-dd'
  M_time        time            format 'hh:mm'
  id            int             format i5
  position      text     8
  status        text     1
tables
  techs with id name position status
  conductors with symbol at_no resistivity name
  measures with symbol id M_date M_time resistivity
links
  M_C from symbol in measures to symbol in conductors
  M_T from id in measures to id in techs
end
\end{verbatim}
\caption{Initial definition of Sample database}
\label{Sample-def1}
\end{figure}
 
 
\begin{figure}[htp]
\begin{verbatim}
define Sample
columns
  notes         text     var    format a25
tables
  notes with symbol id M_date M_time notes
links
  notes_c from symbol in notes to symbol in conductors
end
\end{verbatim}
\caption{Modification of Sample database definition}
\label{Sample-def2}
\end{figure}
 
\section{Accessing a database}
%
\I{open@"open"}
\subsection{"open"}
You access an existing database by opening it with this command
\verb|open <name>|
Some systems allow you to specify file pathnames or
directory names to locate the database files.
Consult Appendix~\ref{UCS-systems} for details.
 
When Rim opens a database it also looks for an input
file with the same name as the database and with an extension
of ".rim".  If this file exists it is assumed to contain
Rim commands and will be input automatically.
\I{initialization}
This initialization file is commonly used to load
database specific macros (See chapter~\ref{mac-chapter}).
 
\I{close@"close"}
\subsection{"close"}
Rim will close an open database as it exits, but you
may also manually close it at any time with the close
command
\verb|close|
 
 
\section{Loading tables}
%
\I{load@"load"|(}
You can load data rows into tables from two types of sources.
 
\begin{List}
\item[Free-format] is the form of most hand-typed data.
  In free-format sources:
  \begin{itemize}
  \item data items are delimited by spaces and commas;
  \item data on the input records are in the same order as
     the columns of the table; and
  \item each input record must exactly fill a table row (The
     input record may, of course, span several lines if each
     is continued with a "+").
  \end{itemize}
\item[Fixed-format] is the form of most computer generated data.
  In fixed-format sources:
  \begin{itemize}
  \item data items are located in fixed columns---no delimiters
     are required;
  \item numeric data may appear anywhere in a field---leading and
     trailing blanks are ignored;
  \item leading blanks in text data are always retained;
  \item trailing blanks in text data are stripped and do not
     affect the length of variable length text columns;
  \item data on the input records need not be in the same order as
     the columns of the table;
  \item each row may span several records (The "+" is not applicable);
  \item some columns of the input records may be ignored;   and
  \item some rows of the table may not be filled (These will become
     ``missing values'').
  \end{itemize}
\end{List}
 
In order to load data, you must have entered
either the owner password
\I{password}
or the modify password (MPW) for the table you are loading.
 
\subsection{Loading free-format data}
%
\I{load@"load"}
This form will load data either from the terminal
or from a file.  The command block
\verb|load| <table>\\
      <data records>\\
      "end" \opt{"load"}

loads the data records into the specified table.
 
If the data records are in a file, use the command
"load" <table> "from" <filename>
to load the table.  In this case the trailing "end" is optional.
Rim will assume the end if it reaches the end-of-file.
See the notes on filenames in appendix \ref{UCS-systems}.
Figure \ref{Sample-loadf} shows the input that loaded
most of the Sample database.
 
\begin{figure}[htp]
\begin{verbatim}
*( Load the Sample database )
open Sample
load conductors
  Cu 29 12.2 Copper
  Al 13 12.9 Aluminum
  Fe 26 20.3 Iron
  LB -NA- -NA- Bernstein
  U 92  -MV- Uranium
end
 
load techs
  5  'John Jones'   'Tech 1'  A
  22 'Jim Smith'    'Tech 2'  A
  35 'Joe Jackson'  'Tech 1'  A
end
 
*(The colon is a Rim delimiter and cannot be
  used in unquoted strings)
*(Since Rim ignores non-essential fields in dates and times,
  a dot is used as the spacer in this fields.)
 
load notes
  Cu  5  88/01/21  09.03  +
  'Spilled coffee on sample.  Expect higher readings due +
to poor electrical contact.'
 
  Cu 22  88/02/22  09.20  +
    'Sample seems contaminated with bitter deposit. +
Recommend addition of cream and sugar.'
end
\end{verbatim}
\caption[Free-form loading of Sample database.]%
{{\centering Free-form loading of Sample database.\par\medskip}
 Note that the input data is in normal Rim form---comments are allowed,
 fields are separated by spaces or commas, strings are enclosed in
 quotes, and lines are continued with a trailing plus sign.}
\label{Sample-loadf}
\end{figure}
 
 
\subsection{Loading fixed-format data}
%
\I{load@"load"}
This form is almost always used with the data in a file.
Fixed-format loading requires a format definition block
that describes the position and format of each
column to be loaded.
It is also usually in a file and looks like
\verb|format|\\
    <line> <position> <column\_name> <format>\\
    \vdots\\
    "end" \opt{"format"}
where <line> is 1 for the first input record of a row, 2 for
the next input record, etc.
<Position> is the starting character position in the record
for the data.  Data formats are described in figure \ref{formats}.
You may enter a null field ("' '") in place of the <column\_name> and 
<format> to cause Rim to skip an input line.
 
 
\begin{figure}[tp]
 
\asistopbox{0}{11pc}
\scriptsize
\begin{verbatim}
 Cu      01/21/88  08:10:00
  5            11.9
 Cu      01/21/88  10:32:00
  5            12.4
 Al      02/10/88  13:45:00
 35            13.0
 Al      02/11/88  09:34:00
 35            14.3
 Cu      02/22/88  08:48:00
 22            12.5
 Fe      03/04/88  15:33:30
  5            19.4
\end{verbatim}
\endasisbox
 
\asistopbox{1}{14pc}
\scriptsize
\begin{verbatim}
*( format file for measures table )
 
format
   1  2  symbol   a8
   1  10 M_date   'mm-dd-yy'
   1  20 M_time   'hh:mm:ss'
   2  2  id       i2
   2 10  resistivity    f7.3
end
\end{verbatim}
\endasisbox
 
  \makebox[\textwidth]{\tt \hskip\parindent
   load measures from mxxxx.dat using measures.fmt \hfill}
 
  \begin{picture}(360,30)
  \put(72,0){\vector(2,1){50}}
  \put(260,0){\vector(-1,1){25}}
  \end{picture}
 
  \makebox[\textwidth]{\fbox{\box0}\hfill \fbox{\box1}}
 
\caption[Fixed-form loading of Sample database.]
{{\centering Fixed-form loading of Sample database.\par\medskip}
 Note that the format file is normal
 Rim form but the data file is not.}
\label{Sample-loadfx}
\end{figure}
 
Begin formatted loading with
"load" <table> "from" <data\_file> "using" <format\_file>
where the data is in file <data\_file> and the format block
is in file <format\_file>. The latter may be ``"terminal"''
if you want to enter the format block online.
If both filenames are the same Rim will first read the
format and then the data from the same file.
See the notes on filenames in appendix \ref{UCS-systems}.
 
Figure \ref{Sample-loadfx} shows an the input that loaded
the "measures" table.
\I{load@"load"|)}
 
 
\section{Retrieving data---"select"}
%
\I{select@"select"|}
This section shows you how to retrieve data stored in
your database.
The "select" command, which has many clauses and optional forms,
is the easiest way to look at your data.
If you need more flexibility, or more complicated reports,
you should use the report writer (Chapter~\ref{report-writer}).
The "select" command has the general format
"select" <column specifications> + \\
  \qquad \opt{"from" <relation>} + \\
  \qquad \opt{<where clause>} + \\
  \qquad \opt{<sort clause>} + \\
  \qquad \opt{"to" <filename>}
where the last three clauses are optional.
If you omit the "from" clause Rim will use the most recently
accessed table.  If there is no such table
Rim will report an error.
 
\subsection{Column specifications}
%
This is a list of the columns to display, along with an optional
description of how to display them.  The column may be either
from the main table (as indicated on the "from" clause) or may
be from a linked\I{link} table.
In its simplest form a column specification is just a list of columns.
 

\begin{verbatim}
select id m_date from measures
 id     M_date
 -----  --------
     5  88-01-21
     5  88-01-21
    35  88-02-10
    35  88-02-11
    22  88-02-22
     5  88-03-04
\end{verbatim}
 
\subsubsection{Including columns from linked tables}
%
\I{link}
If the column to be displayed is in a linked table, specify
the link name first, followed by a colon and then the column name.
<link\_name>:<column\_name>
where <column\_name> refers to a column in the linked table.
 

 \begin{verbatim}
select symbol id M_t:name M_date from measures
 symbol    id     name          M_date
 --------  -----  ------------  --------
 Cu            5  John Jones    88-01-21
 Cu            5  John Jones    88-01-21
 Al           35  Joe Jackson   88-02-10
 Al           35  Joe Jackson   88-02-11
 Cu           22  Jim Smith     88-02-22
 Fe            5  John Jones    88-03-04
 \end{verbatim}
 
 
\subsubsection{Specifying formats}
\I{format}
\label{format-spec}
You may specify a format for each column if you don't want to use
Rim's default. After the column name type an ``at'' sign ("@") and the format.
<column\_name>"@"<format>
The format syntax is described in figure \ref{formats}.
 
 
If you do not specify a format Rim will use a default
for the column---as specified either by "define" commands or by the
\I{define@"define"}\I{reformat@"reformat"}
"reformat" command.  If neither of those was specified Rim will
use a default format according to the column type.
You may display these defaults with the "show" command (page \pageref{show})
and may set them with the "set" command (page \pageref{set}).
\I{show@"show"}\I{set@"set"}
 
\label{sel-dem1}
\begin{verbatim}
  select symbol id@i4 M_t:name
  m_date@'dd-mmm-yy' from measures
 symbol    id    name          M-date
 --------  ----  ------------  ---------
 Cu           5  John Jones    21-JAN-88
 Cu           5  John Jones    21-JAN-88
 Al          35  Joe Jackson   10-FEB-88
 Al          35  Joe Jackson   10-FEB-88
 Cu          22  Jim Smith     22-FEB-88
 Fe           5  John Jones    04-MAR-88
\end{verbatim}
 
 
\subsubsection{Specifying titles}
\I{title}
You may specify the title for each column if you don't want to use
Rim's default. After the column name type a percent sign and the title.
<column\_name>"\%"<title>
<Title> should be enclosed in quotes.
 
\begin{verbatim}
select symbol id\%'Tech id' M_t:name  
    m_date@'dd-mmm-yy'\%'Measure date' from measures
 symbol    Tech id  name          Measure date
 --------  -------  ------------  ------------
 Cu            5    John Jones    21-JAN-88
 Cu            5    John Jones    21-JAN-88
 Al           35    Joe Jackson   10-FEB-88
 Al           35    Joe Jackson   11-FEB-88
 Cu           22    Jim Smith     22-FEB-88
 Fe            5    John Jones    04-MAR-88
\end{verbatim}
 
\subsubsection{Summing columns}
\I{sum}
You may total a column by following it with an equal sign and an `S'.
<column\_name>"=S"
sums the column.
There is no useful summation in the Sample database
to use for an example.
 
\subsection{Selecting all columns}
\I{*}
You may select all columns of a table, with the default formats
and titles, with an asterisk ("*") in place of the column specifications.
\begin{verbatim}
select * from measures
 symbol    id     M_date    M_time   resistivity
 --------  -----  --------  -------  ------------
 Cu            5  88-01-21  08:10     11.900
 Cu            5  88-01-21  10:32     12.400
 Al           35  88-02-10  13:45     13.000
 Al           35  88-02-11  09:34     14.300
 Cu           22  88-02-22  08:48     12.500
 Fe            5  88-03-04  15:33     19.400
\end{verbatim}
 
\subsection{Function specifications}
\I{functions}
Rather than selecting individual rows from a table, you can
have Rim perform simple tallies on columns.
The general form of the function "select" is
"select" \opt{<column>} <function>"("<column> \opt{\ldots}")"
   \opt{\ldots} "from" <table>
where the functions may be
"num()  sum()  ave()  max()  " or " min()"
each with an obvious functionality.
Only one detail column may be specified.  If it is omitted
the selection will produce a single tally for the table.
Functions will always completely disregard any missing values.

\begin{verbatim}
  select symbol m_c:name num(resistivity)
      ave(resistivity@f9.2)  from measures
 symbol    name          NUM(resistivity )  AVE(resistivity )
 --------  ------------  -----------------  -----------------
 Al        Aluminum             2               13.65
 Cu        Copper               3               12.27
 Fe        Iron                 1               19.40
\end{verbatim}
 
\subsection{Where clause}
%
\I{where@<where clause>}
\label{where-clause}
The "where" clause acts as a filter which passes only
those rows that pass the clause criteria.
For example,
"select" <columns> "from measures where symbol='Cu'"
selects only the measurements of copper (Cu) and ignores
the other rows.
The "where" clause consists of
 
\begin{List}
 \item[comparisons] of the form
   <column> <op> <value>\\
                <column> <op> <column>\\
                <column> <test>
  where <op> and <test> are described in figure~\ref{op-test},
 
\begin{figure}[p]
  \begin{center}
  \begin{tabular}{ccp{19pc}}
    op & alt& $V=A\;op\;B$\\
    \hline
    eq &  "\tt ="& V is true if $A=B$\\
    ne &  "\tt \char`\<>"& V is true if $A\ne B$.\\
    ge &  "\tt >="& V is true if $A\ge B$.\\
    gt &  "\tt >" & V is true if $A>   B$.\\
    le &  "\tt \char`\<="& V is true if $A\le B$.\\
    lt &  "\tt \char`\<" & V is true if $A<   B$.\\
    like&&      V is true if $A\in B$.
                     (See note 1)\\
  \noalign{\bigskip}
    test&&  $V=A\;test$\\
    \hline
    exists&&V is true if A is defined.\\
    fails&&V is true if A is undefined.
 A missing value or a not applicable value is undefined. All
 other values are defined.
   \end{tabular}
   \end{center}
 
 \I{like}\I{text matching}
 \begin{description}
 \item[1)] The "like" comparison
    is for strings only.  The strings are compared
    character-by-character, however
    the B string for this comparison may contain `wild-card'
    characters.
    A `"?"' in B will match any single character in A, while
    an `"*"' in B will match any number of characters in A.
    The `wild-card' characters
    may be changed with the SET ARBCHAR command.  This comparison may be used
    only with a value---<column> "like" <column> is not valid.
 
 \item[2)] String matching is also sensitive to the setting of the
 "case" setting, which determines whether or not upper
 and lower case letters match one another.  The default
 is "case respect" which says that upper case letters
 do \underline{not} match lower case letters.
\end{description}
\caption{"where"\ clause comparisons}
\label{op-test}
\end{figure}
 
 \item[boolean operators] which combine comparisons
   <comparison> "and" <comparison> \\
                <comparison> "or " <comparison> \\
           \phantom{<comparison>} "not" <comparison>
 
 \item[parentheses] which specify the precedence
   of the boolean operations
 
     \begin{tabular}{cl}
     $(C_1\; op_a\; C_2) op_b\; C_3$&
        $C_1\; op_a\; C_2$, then $ op_b\; C_3$\\
     $C_1\; op_a\; (C_2\; op_b\; C_3)$&
        $C_2\; op_b\; C_3$, then $ C_1\; op_a$
     \end{tabular}
\end{List}

\begin{verbatim}
select symbol m_c:name resistivity@f7.2 from measures
    where symbol='Cu'
 symbol    name          resistivity
 --------  ------------  ------------
 Cu        Copper          11.90
 Cu        Copper          12.40
 Cu        Copper          12.50
\end{verbatim}
 
\subsection{Sort clause}
%
\I{sort@<sort clause>}
\label{sort-clause}
The "sort" clause specifies the sort order of the displayed
rows.
It's form is
"sort" \opt{"by"}
  <column> \Opt{\stk{"=A"\\"=D"}} \opt{\ldots}
where "=A" indicates an ascending sort order (the default)
and "=D" indicates a descending sort order.
The first column in the clause is the primary sort column;
the second is the secondary sort column; etc.
 
A "sort" clause may not be specified with a function "select"---the
independent column is necessarily the sort column.
 
\begin{verbatim}
select symbol m_c:name resistivity from measures
    sort by symbol
 symbol    name          resistivity
 --------  ------------  ------------
 Al        Aluminum       13.000
 Al        Aluminum       14.300
 Cu        Copper         11.900
 Cu        Copper         12.400
 Cu        Copper         12.500
 Fe        Iron           19.400
\end{verbatim}
\I{select@"select"|)}
 
\section{Making Retrievals Faster---"Keys"}
%
\I{key|(}
This section shows you how to speed up your data retrievals
by building keys for some commonly accessed columns.
A key is a list of pointers to the data records
of a table. The key is ordered by the values of the column
for which the key is defined.
Whenever Rim is performing an ``equals'' type of search,
it will use the pointers to access data records, resulting in
a much faster retrieval.
However, updates to the table are more time consuming when
columns are keyed.  When possible, you should first load
the table and then build any useful keys.
 
\subsection{Building keys}
%
\I{build key@"build key"}
Build a key for a column with this command.
"build key for" <column> "in" <table>
 
Usually, columns that are targets of links will be
keyed columns.
The Sample database should have these columns keyed:
" build key for symbol in conductors"\\
             "build key for id     in techs     "
 
In addition,
if the "notes" table becomes large, it might be reasonable
to build keys for both "symbol" and "id" in "notes".
 
\subsection{Removing keys}
%
\I{remove key@"remove key"}
Remove a key for a column with this command.
"remove key for" <column> "in" <table>
\I{key|)}
 
 
\section{Changing data values}
%
\I{change@"change"}
This section shows you how to change the data values in your tables.
Use the change command:
"change" <column> "to" <new\_value> "in" <table>
           "where" <where\_clause>
which changes the value of <column> to <new\_value> for every
row of <table> that satisfies the "where" clause.
 
\section{Deleting data rows}
%
\I{delete rows@"delete rows"}
This section shows you how to delete rows from your tables.
Use the delete command:
"delete rows from " <table> "where" <where\_clause>
which deletes all rows in <table> that
satisfy the "where" clause.
 
\section{Unloading your database}
%
\I{unload@"unload"}
If you want to transfer your database to another computer (which
supports Rim), or if you want to save it in a system-inde\-pend\-ent
form you can "unload" it to a text file.  You may unload
all or selected tables, and either
the definitions, the data, both, or the passwords.
Use the command
"unload" \Opt{\stk{"all"\\"definitions"\\"data"\\"passwords"}}
   \opt{<table> ...} \opt{"to" <filename>}
to unload your database.  The default is to unload both definitions
and data (all) and to unload all tables.
The unloaded file is in normal text format with lines no longer
than 80 column per line.
 
You can reload the unloaded database simply by inputting the
text file into Rim.
 
\medskip
 
\I{reload}
Notice that this command also gives you a means to reload your database.
You might want to do this when it accumulates a large number of deleted
records.
 
To reload your database:
\begin{enumerate}
\item Unload the database to a text file.
\item Backup and purge the database files.
\item Input the text file to Rim to rebuild the database.
\end{enumerate}
 
 
\section{Making new tables from old - relational algebra}
%
\label{ra-section}
\I{algebra|(} \I{relational database}
You can use relational algebra on tables in your database
to create new tables. This section describes Rim's
relational algebra commands.
 
\subsection{Union}
%
\I{union@"union"}
The "union" command creates a new table which contains all
or selected columns from two source tables.  The resultant
table contains all rows from each source table.
When rows from the two tables
match in all common columns, a single resultant row is created.
When rows from the two tables do not
match in all common columns, a resultant row is created
for each source row, with missing values filling out the rows.
\I{missing value}
"union" <table-1> "with" <table-2> "forming" <table-3> "+"\\
  \qquad  \opt{"using" <column-1> <column-2> \ldots }
creates <table-3> from the union of <table-1> and <table-2>.
 
\subsection{Intersection}
%
\I{intersection@"intersection"}
The "intersection" command creates a new table which contains all
or selected columns from two source tables.  The resultant
table contains only those rows from each source table
which
match in all common columns.
"intersect" <table-1> "with" <table-2> "forming" <table-3> "+"\\
  \qquad  \opt{"using" <column-1> <column-2> \ldots }
creates <table-3> from the intersection of <table-1> and <table-2>.
 
\subsection{Subtraction}
%
\I{subtract@"subtract"}
The "subtract" command creates a new table which contains all
or selected columns from two source tables.  The resultant
table contains only those rows from the second table
which
do not match all common columns of the first table.
"subtract" <table-1> "from" <table-2> "forming" <table-3> "+"\\
  \qquad  \opt{"using" <column-1> <column-2> \ldots }
creates <table-3> from the subtraction of <table-1> from <table-2>.
 
\subsection{Join}
%
\I{join@"join"}
The "join" command creates a new table which contains all
columns from two source tables.  The resultant
table contains rows from each source table
where specified columns match according to the specified comparison.
"join" <table-1> "using" <column-1>
    "with" <table-2> "using" <column-2> "+"\\
  \qquad  "forming" <table-3>
    "where" <comparison> 
creates <table-3> from the join of <table-1> and <table-2>.
Each combination of rows where
<column-1> in <table-1> \qquad <comparison> \qquad <column-2> in <table-2>
is true, creates a row in <table-3>.
 
\subsection{Projection}
%
\I{project@"project"}
The "project" command creates a new table which contains all
or selected columns from a source table.  The resultant
table contains all or selected rows from the source table.
"project" <table-1> "from" <table-2>
  \opt{"using" <column-1> <column-2> \ldots } "+"\\
 \qquad   "where" <conditions> 
creates <table-1> from <table-2>.
<table-1> contains selected columns ("using") and the
selected rows ("where") of <table-2>.
%
\I{algebra|)}
 
 
\section{Redirecting input and output}
%
\I{set!input@"input"}
\I{input@"input"}
\I{filename}
Rim normally read commands from your terminal and writes back
to your terminal.  You can tell Rim to read from a file by
entering the command
\opt{"set"} "input" <filename>
where <filename> identifies the file to read.  It is specified
in the standard format for your system.
On systems which commonly use two-part filenames an extension
will be assumed if it is not specified.
Consult Appendix~\ref{UCS-systems} to
see if Rim will assume an extension for your input files.
An `end-of-file' generates an automatic "end" command.
 
\I{set!output@"output"}
\I{output@"output"}
\I{filename}
You can redirect Rim's terminal output to a file by
entering the command
\opt{"set"} "output" <filename>
where <filename> identifies the file to write to.
On systems which commonly use two-part filenames an extension
will be assumed if it is not specified.
Consult Appendix~\ref{UCS-systems} to
see if Rim will assume an extension for your output files.


\section{Showing and setting Rim's parameters}
%
\subsection{Showing}
Rim is governed by many parameters.  Many of these represent
fixed limits.  You can see what these limits are with
this command
\I{show@"show"}
"show limits"
 
You have direct control over many other parameters.
You can see these with
this command
"show"
\label{show}
 
You can also use the "show" command to view macro definitions
(see section~\ref{show-mac} on page~\pageref{show-mac}).
 
\subsection{Setting}
The parameters you may set are:
%
\label{set}
\I{set!name@"name"}
\I{database name}
"set name" <new name>
sets the database name to <new name>.
This is a permanent change to the database.
%
%\I{set!owner@"owner"}
%\I{password}
%\<"set owner" <password>\>
%sets the database owner password to <password>.
%This is a permanent change to the database.
%
\label{ucmd-section}
\I{set!user@"user"}
\I{user@"user"}
\opt{"Set"} "user" <password>
sets the user password.
%
\label{echo-mode}
\I{set!echo@"echo"}
\I{echo@"echo"}
\opt{"Set"} "echo" \Opt{\stk{"on"\\"off"}}
sets the command echo mode.  When echoing is on, all commands
are `echoed' to the terminal, or to the report file if
output has been redirected.
%
\I{set!case@"case"}
\I{text matching}
"Set case" \stk{"ignore"\\"respect"}
sets the case mode for text matching.
If case is "ignore",
lower and upper case letters will match.
If case is "respect",
lower and upper case letters will not match.
%
\I{set!single@"single arbchar"}
\I{text matching}
"Set single arbchar" <character>
sets the `single' wild-card character for "like" string matching.
Occurrences of <character> in the template string will match
any single character in the target string.
The default value for this character is a question mark ("?").
%
\I{set!multiple@"multiple arbchar"}
\I{text matching}
"Set multiple arbchar" <character>
sets the `multiple' wild-card character for "like" string matching.
Occurrences of <character> in the template string will match
any number of characters (including none) in the target string.
The default value for this character is an asterisk ("*").
%
\I{set!mv@"MV"}
\I{mv@"MV"}
\I{missing value}
"Set MV" <string>
sets the "MV" missing value string for data input and output.
"MV" missing values will appear in reports as <string>.
Occurrences of <string> in input data will load the corresponding
column with a type "MV" missing value.
If the <string> is blank, any blank input field will
be assigned the MV missing value.
The default value for MV is `{\tt-MV-}'.
 
 
%
\I{set!na@"NA"}
\I{na@"NA"}
\I{missing value}
"Set NA" <string>
sets the "NA" missing value string for data input and output.
"NA" missing values will appear in reports as <string>.
Occurrences of <string> in input data will load the corresponding
column with a type "NA" missing value.
If the <string> is blank, any blank input field will
be assigned the NA missing value.
The default value for NA is `{\tt-NA-}'.
%
\I{set!formats@<formats>}
\I{format}
"Set" \stk{"integer"\\"real"\\"date"\\"time"}
 "format" <format>
sets the default format for the selected item.
%
\I{set!terminal width@"terminal width"}
"Set terminal width" <number>
sets the width of the terminal to <number> characters.
Lines output to the terminal will not be longer than this value.
%
\I{set!report width@"report width"}
"Set report width" <number>
sets the maximum width of reports to <number> characters.
\I{set!report height@"report height"}
\label{set-height}
"Set report height" <number>
sets the page height for reports to <number> lines.
If <number> is zero, there will be no pagination---%
no headers, no footers, and no forms control.
 
\I{set!trace@"trace"}
\label{set-trace}
You can maintain a log of your commands by "tracing" them to
a file.
"Set trace" \stk{\opt{"to" <file>}\\"off"}
turns tracing on or off.\footnote{Tracing is actually a more
robust feature than is described here.  Is is used to
diagnose Rim problems.  See the {\it Rim Installers Manual}
for more detailed information.}



\subsection{Setting parsing parameters}
%
\label{cmt-cmd}
\I{set!parsingj<parsing parameters>}
\I{parsing}
\I{comments}
You may also change some of the rules Rim uses to parse your
input.  Since you must be able to enter special characters,
Rim requires you to put these commands inside `comments'.
The parsing options are specified as
"*(set " <option>"="<character> ")"
where the <character> is any character or "null",
and <option> is one of the following.
\begin{description}
\item[del] sets the alternate field delimiter (other than space).
           If it is set to null, space is the only delimiter.
           The default is a comma (",").
 
\item[con] \I{continuation}
           sets the line continuation character.  When this character
           appears at the end of an input line the command
           is continued on the next line.
           If it is set to null, all lines are continued and
           a command ends only at the "end" character.
           The default is a plus ("+").
\item[end] sets the end of command character.  Where this character
           appears in the input line the command is completed.
           The next command may follow on the same line.
           If it is set to null, all commands must end at the
           end of a line.
           The default is a semicolon (";").
\end{description}
For example,
" *(set del=/) *(set con=null)"
indicates that the virgule (/) will separate fields and that all
commands will end only at semicolons.
 
 
\section{Recalling commands}
%
\I{commands}
\I{r@"r"}\I{recall}
You may want to recall, edit, and reuse a previous command,
either to fix an error or to add or delete parameters.
Or you may want to use an old command as a basis for
construction of a new command.  Entry of a single
"r"
on the command line will recall the most recently entered
command {\em line} for edit and execution.
 
When the recalled command is displayed, you may
 
\begin{itemize}
\item enter another `"r"' to recall the next most recent command,
\item enter \fbox{\small return} to execute the command, or
\item edit the command.  Characters typed replace corresponding
  characters in the command, except that
 
  \begin{description}
  \item[<space>] retains a character,
  \item["\tt \#"] deletes a letter,
  \item["\tt \char`"] begins insertion of characters,
  \item["\tt >"] ends insertion of characters, and
  \item["\tt !"] truncates the line.
  \end{description}
 
  The edited command will be displayed and you may make further
  edits or execute it.
 
\end{itemize}
 
%
%  <<<<< Macros >>>>>>>>>>
%
\chapter{Defining and Using Macros}
\I{macros|(}
\label{mac-chapter}
A macro is a word that has the meaning of several words.
More specifically, a macro has a name and a definition.
When Rim encounters a macro name during text input, it
replaces the name with the macro's definition.
 
Define a macro with the command
\I{macro@"macro"}
"macro <name> = <definition>"
where
 \begin{tabular}{lcl}
   <name> &=& name of the macro\\
   <definition> &=& replacement text
   \end{tabular}
 
\section{Simple macros}
Here is an example which illustrates macro usage.
 
Define the macro "person" with the command
"macro person = 'id@i4 m\_t:name'"
Then you can use the name "person" in selection commands, such as
this example.\footnote{Compare this to the command on
page~\pageref{sel-dem1}.}
 
\I{select}
\begin{verbatim}
  select symbol person 
      m_date@'dd-mmm-yy' from measures
 symbol    id    name          M-date
 --------  ----  ------------  ---------
 Cu           5  John Jones    21-JAN-88
 Cu           5  John Jones    21-JAN-88
 Al          35  Joe Jackson   10-FEB-88
 Al          35  Joe Jackson   10-FEB-88
 Cu          22  Jim Smith     22-FEB-88
 Fe           5  John Jones    04-MAR-88
\end{verbatim}
 
Rim has interpreted "person" to mean
"id@i4 m\_to\_t:name".
 
\section{Macros with Arguments}
The "person" macro in the previous example provides a convenience,
but has limited utility.  The macro's meaning never changes.
Often you want a slightly different meaning
each time the macro is invoked.
 
To gain this flexibility define a macro
with arguments.
\I{argument}
The following example demonstrates such a macro.
 
Define the macro "rename" by the command
"macro rename = 'change name to \char`\"' 2 '\char`\"\ in techs ' +"\\
   \qquad " 'where id = ' 1 "
\label{rename-def}
 
With this definition
you can use the new `command' "rename"%
   \footnote{ Even with the above definition of the macro
   {\bf rename}, you can
    still use Rim's {\bf rename} command to rename tables, links, or
   columns.  You just have to abbreviate the command.
   Rim will only recognize a macro when its name
   is typed in full.}
to change names in
the "techs" table. For example,
" rename 5 'John A. Jones' "
which Rim will interpret as
" change name to \char`\"John A. Jones\char`\"\ in techs +"\\
   \qquad " where id = 5 "
 
\section{Macro expansion}
The macro definition text consists of text fields
and integer argument numbers (range 1--31).
When the macro is invoked,
by the occurrence of its name in input text, the text portions
of the macro's definition are copied verbatim.  The integers
in the definition are replaced by corresponding argument
fields following the macro's name in the input.
No spaces are added between fields in the definition text.
For example, with the definition
" macro alfa = 'abc ' 2 ' def' 1 '; xxx'"
the input
" select alfa this that from ..."
is interpreted by Rim as
" select abc that defthis; xxx from ..."
 
Note that there were no spaces between `"def"' and the argument number ("1").
If you want separation spaces around the arguments, as with argument
"2" here, you must put them in.
 
Note also that this macro ("alfa") contained an end-of-line character
(";").  Macros may expand into several Rim commands.
 
If the "alfa" macro had been invoked with fewer arguments, for example,
" select alfa this"
it would be interpreted by Rim as
" select abc  defthis; xxx from ..."
The unused arguments are simply ignored.
 
A more sophisticated macro example is discussed
in the report writing chapter, on page~\pageref{run-mac}.
 
\section{Looking at your macro definitions}
\label{show-mac}
Macros are expanded prior to being "echo"d---assuming "echo"
has been set.
You can look at the expansion of any macro by "echo"ing a
comment command ("*").  For example, assuming the "rename" 
definition on page~\pageref{rename-def}, you could enter
"set echo"\\"* rename x Anyone"
and Rim would echo
" change name to Anyone in techs where id = x "
   

You can also look at any or all of your macro definitions
with this "show" command
"show macro \opt{<name>}"
If <name> is given, only that macro will be displayed.
 
\section{Clearing macro definitions}
You clear (delete) a macro definition with the command
"macro <name> clear"
Clearing macros recovers macro definition space.
\I{macros|)}
 
 
%
%  <<<<< Report Writer >>>>>>>>>>
%
\def\demobreak{\par\pagebreak[3]\bigskip}

%
\chapter{Printing Reports}
%
\label{report-writer}
\I{Reports|(}
Rim's flexible report writer lets you produce
sophisticated reports on your database.
You have complete control over page headers, footers,
and detail lines.  You can combine data from several tables
in a single report. And you can alter the format of the
report based on the actual content of individual rows
in the database.
 
Before learning to use the report writer you should be
familiar with the Rim commands described in the previous chapter.
In particular, you must
thoroughly understand the "select" command.
 
\section{How a report is generated}
%
A Rim report is the product of several commands, which are first
`compiled' by Rim and then `executed' as a block to produce the report.
You therefore rarely actually type report writing commands
directly into Rim.  Instead, you should enter them into a separate
text file---using the local system editor---and then "input" them
to produce the report.
\I{input@"input"}
 
 
Here is an example input file and the report it produces to
illustrate Rim's report writing statements.
\demobreak
\begin{verbatim}
report
  header
    1 1 'symbol'
    1 10 'date'
    1 20 'resistivity'
    2 1 ' '
  end header
  select from measures sort by symbol
    print
      1 1 symbol a8
      1 10 m_date 'yy/mm/dd'
      1 20 resistivity f8.2
      2 1 ' '
    end print
  end select
end report
\end{verbatim}
 
\begin{verbatim}
 symbol   date         resistivity
 
 Al       88/02/10     13.00
 Al       88/02/10     14.30
 Cu       88/01/21     11.90
 Cu       88/01/21     12.40
 Cu       88/02/22     12.50
 Fe       88/03/04     19.40
\end{verbatim}
 
Notice first that the report definition consists of blocks,
which begin with a `<command>' and end with `"end" <command>'.
The indentation is intended to accentuate this block structure
and you are advised to follow this practice in your own reports.
 
 
In this example, the "select" block identifies the rows and sort
order of the rows of the
<measures> table that are to be extracted (this example selects all rows).
The "print" block
describes the actual format of data on the report.
The "report" block contains the entire report definition.
 
\smallskip
\I{statement}
Report writer commands are called
as {\em statements} to distinguish them from other Rim commands.
 
\section{Variables}
%
\I{variable}
The report writer allows you to define variables, assign values to
them, and use them much like columns of a table.
\I{assignment statement}
The assignment statement
<variable> \Opt{\stk{"int"\\"real"\\"double"\\"date"\\"time"\ \# >}} "="
     <expression>
defines <variable> (if it hasn't yet been defined), gives it
a type ("int" is the default, <\#> is the length of a text variable),
and assigns it the value of <expression>.
Variable names are similar to column names.
 
\I{expression}
The <expression> is any rational (to Rim) combination of values,
table columns, and variables connected by the
mathematical operators "+", "-", "*", and "/", with parenthesis
allowed to specify operator precedence.
This is much like the assignment statement of any common programming
language. For example,
 
\begin{verbatim}
     temp_id = id
     xsym 8 = symbol
     next_day = m_date + 1
     ave_error = (tot_resist - ave_resist) / count
\end{verbatim}
 
are valid assignment statements.  These rules govern
expressions.
 
\begin{enumerate}
\item Text data may be concatenated with the + operator.
\item Vector and matrix elements may be specified
   by <name>"("<item>")" or <name>"("<row> <column>")"
   in standard Rim fashion.
      \I{vector}\I{matrix}
\item If both operands are integers, the arithmetic
   and the resultant will be integer.
\item If either operand is real or double, the arithmetic
   and the resultant will be double precision.
\item The default operator precedence is left to right
   in all cases.  For example,
   "a+b*c" \qquad $\equiv$ \qquad  "(a+b)*c".
\item When a real value is converted to an integer, the value is truncated.
      \I{real}\I{integer}
\item String values are either padded with blanks or truncated
   to match the length of the receiving variable.
\item Date and time columns are treated as integers.
      \I{date}\I{time}
\item Any occurrence of ``missing values'' in an expression
   gives the expression a ``missing value''.
      \I{missing value}
\item There are four pre-defined variables:
\begin{List}
  \item["page\_number"] -- the number of the current page,
      \I{page\_number@"page\_number"}
  \item["line\_number"] -- the number of the current line,
      \I{line\_number@"line\_number"}
  \item["report\_date"] -- the date at the start of report execution, and
      \I{report\_date@"report\_date"}
  \item["report\_time"] -- the time at the start of report execution.
      \I{report\_time@"report\_time"}
\end{List}
\end{enumerate}
 
\section{"report"}
%
\I{report statement}
The report block defines the entire report.
"report"\\ \quad\vdots\\"end" \opt{"report"}
 When Rim sees the
"report" statement, it begins compiling the report.  When
it sees the "end report" statement, it begins to process the stored commands.
Any error during the compilation phase will terminate compilation
and the report will not be printed.
 
 
\section{"select"}
%
\I{select statement}
The selection block identifies a table, selection criteria for
selecting rows from that table, and a sort order for row selection.
The "select" statement is very similar to Rim's "select" command
except that it contains only the "from", "where", and "sort" clauses.
"select from" <table> \opt{"where" <criteria>}
  \opt{"sort" \opt{"by"} <sort order>} \\
  \quad\vdots\\"end" \opt{"select"}
where
\begin{List}
\item[<table>] may be any table in your currently open database,
\I{table}
\item["where ..."] is a normal, valid Rim "where" clause
(see section~\ref{where-clause}) that may also contain variable
names in addition to values and column names from <table>, and
\I{where@<where clause>}
\item["sort ..."] is a normal, valid Rim "sort" clause
(see section~\ref{sort-clause}).
\I{sort@<sort clause>}
\end{List}
 
All statements within the selection block,
which may include other "select" statements,
will be processed for each row of <table>
that passes the "where" criteria.  Rows are retrieved
and processed in the order specified by the "sort" clause.
For example, the statements
 
\begin{verbatim}
   yesterday = report_date - 1
   select from measures where M_date = yesterday
      .
      .
      .
   end select
\end{verbatim}
 
selects all measurements from the day before the report date.
 
 
\section{"print"}
%
\I{print statement}
Data are actually printed on the report by print blocks.
Print blocks specify the data to be printed, their formats,
and their positions within the block.
"print"\\
  <line> <position> \stk{<column> \ variable>} <format> \\
  <line> <position> <text\_string>\\
    \vdots\\
  "end" \opt{"print"}
Each print block produces a specific number of lines---which is at least
the maximum of the <line> values, but may be greater if an item is paragraphed.
 
\begin{List}
\item[<line>] is the relative line number within the block on which
  this item is to begin printing.
  The first line is 1.
\item[<position>] is the horizontal position on the line
  to begin printing this item.
  The first position is 1.
\item[<format>] is a valid format (see section~\ref{format-spec}).
  \I{format}
\end{List}
 
Paragraphed items will continue on following lines.
Make be sure you don't print anything directly beneath a paragraphed item.
\I{paragraph text}
 
\demobreak
This report program
\begin{verbatim}
report
  select from notes sort by id
    print
      1 5 id     i2
      2 5 symbol a8
      1 15 m_date 'yy/mm/dd'
      2 15 m_time 'hh:mm'
      1 25 notes   a20
    end print
    print       *(print a blank line)
      1 1 ' '
    end print
  end select
end report
\end{verbatim}
\demobreak
produces this report
 

\begin{verbatim}
      5        88/01/21  Spilled coffee on
     Cu        09:03     sample.  Expect
                         higher readings due
                         to poor electrical
                         contact.
 
     22        88/02/22  Sample seems
     Cu        09:20     contaminated with
                         bitter deposit.
                         Recommend addition
                         of cream and sugar.
\end{verbatim}
\label{para-demo}
 
 
\section{Nested selections}
%
\I{select statement}
One of the things a selection block may contain is another
selection block.
The inner block is processed completely for {\em each} selected
row of the outer block.
Column names always refer only to the `current' table---column
names from other tables are inaccessible.  All defined
variables are always available.
 
 
In programming terminology we say that column names are `locally'
defined and variables are `globally' defined.
 
\demobreak
This report program
\begin{verbatim}
report
  select from techs where position = 'Tech 1'
    print
      1 1 id     i4
      1 10 name  a20
    end print
    mid = id  *(save id for measures selection statement)
    select from measures where id = mid sort by symbol
      print
        2 5 symbol a8
        2 15 m_date 'yy/mm/dd'
        2 25 resistivity f6.2
      end print
    end select
    print
      1 1 ' '
    end print
  end select
end report
\end{verbatim}
\demobreak
produces the following report
 

\begin{verbatim}
    5     John Jones
 
     Cu        88/01/21   11.90
     Cu        88/01/21   12.40
     Fe        88/03/04   19.40
 
   35     Joe Jackson
 
     Al        88/02/10   13.00
     Al        88/02/10   14.30
 
\end{verbatim}
 
\section{Headers and footers}
%
\I{header statement}
\I{footer statement}
A header is text printed at the top of each page of a report.
A footer is text printed at the bottom of each page of a report.
They are printed if, and only if,
the report height parameter is non-zero and
the header (or footer) block has been `executed'.
Following are definitions of the "header" and "footer" blocks.
 
"header"\\
  <line> <position> <text\_string>\\
  <line> <position> <variable> <format>\\
  \qquad  \vdots\\
  "end" \opt{"header"}
and
"footer"\\
  <line> <position> <text\_string>\\
  <line> <position> <variable> <format>\\
  \qquad  \vdots\\
  "end" \opt{"footer"}
<Line>, <position>, and <format> have the same meaning
as in the print blocks.
 
\I{set!report height@"report height"}
The report height (see pg.~\pageref{set-height}) refers to
a number of lines of "print"ed lines.  It does not
include header or footer text.
 
\demobreak
This report program
\begin{verbatim}
report
  header
    1 1 '  Id'
    1 10 Name
    1 60 'Page: '
    1 66 page_number i5
    2 1 ' '
  end header
 
  select from techs where position = 'Tech 1'
    print
      1 1 id     i4
      1 10 name  a20
    end print
    mid = id
      print
        2 5 'symbol    date      resistivity'
        3 5 '------    ----      -----------'
      end print
    select from measures where id = mid sort by symbol
      print
        1 5 symbol a8
        1 15 m_date 'yy/mm/dd'
        1 25 resistivity f6.2
      end print
    end select
    print
      1 1 ' '
    end print
  end select
end report
\end{verbatim}
\demobreak
produces the following report
 
\begin{verbatim}
   Id     Name                                              Page:     1
 
    5     John Jones
 
     symbol   date      resistivity
     ------   ----      -----------
     Cu        88/01/21   11.90
     Cu        88/01/21   12.40
     Fe        88/03/04   19.40
 
   35     Joe Jackson
 
     symbol   date      resistivity
     ------   ----      -----------
     Al        88/02/10   13.00
     Al        88/02/10   14.30
\end{verbatim}
 
 
 
\section{Conditional processing---"if"}
%
\I{if statement}
\I{else statement}
You may want to vary the content or format of your report
based on actual data in table columns (or variables).
To do so, use "if" statements.
"if" <test>\\
  --- statements executed if <test> is true ---\\
  "else"\\
  --- statements executed if <test> is false ---\\
  "end" \opt{"if"}
where
\begin{List}
\item[<test>] is a normal "where" clause (without the
``where'').  It may contain columns from the current table,
variables, and values.
\end{List}
 
The <test> is evaluated.
If it is true,
the statements following "if" are executed and the statements
following "else" are skipped.
If it is false,
the statements following "if" are skipped and the statements
following "else" are executed.
The "else" and following statements are optional.
 
\demobreak
This report program
\begin{verbatim}
*(    Rim report of resistance measurements by conductor )
report
 
  header
    1  5 'Resistance measurements by conductor'
    1 60 'Page: '
    1 66 page_number i5
    2 1 ' '
    3 1 'Symbol    No.    Name         Resistivity'
    4 1 ' '
  end
 
  select from conductors
    print
      1 1 symbol a8
      1 10 at_no i4
      1 16 name a15
      1 32 resistivity f6.2
    end print
    xsym 8 = symbol
    mline = 0
    mrest real = 0
 
    rem  select all measurements for this conductor
 
    select from measures where symbol = xsym
      mline = mline + 1
      if mline = 1  *(print the sub-header)
        print
          2 5 '   id   date               resistivity'
          3 5 '-----  --------            -----------'
        end print
      end if
      print
        1 5 id i5
        1 12 m_date 'yy/mm/dd'
        1 32 resistivity f6.2
      end print
      mrest = mrest + resistivity
    end select
 
    rem print a summary for each conductor
 
    if mline > 0
      mrest = mrest / mline
      print
        2 10 'Measured resistivity = '
        2 32 mrest f6.2
        3 1 ' '          *( one blank line following summary)
      end print
    else
      print
        2 10 'No measurements taken'
        3 1 ' '
      end print
    end if
 
  end select
 
end report
\end{verbatim}
 
prints the following report

\begin{verbatim}
     Resistance measurements by conductor                   Page:     1
 
 Symbol    No.    Name         Resistivity
 
 Cu         29  Copper            0.12
 
        id   date               resistivity
     -----  --------            -----------
         5  88/01/21             11.90
         5  88/01/21             12.40
        22  88/02/22             12.50
 
          Measured resistivity = 12.27
 
 Al         13  Aluminum          0.18
 
        id   date               resistivity
     -----  --------            -----------
        35  88/02/10             13.00
        35  88/02/10             14.30
 
          Measured resistivity = 13.65
 
 Fe         26  Iron              0.24
 
        id   date               resistivity
     -----  --------            -----------
         5  88/03/04             19.40
 
          Measured resistivity = 19.40
 
 LB       -MV-  Bernstein         -MV-
 
          No measurements taken
 
 U          92  Uranium           0.44
 
          No measurements taken
 
\end{verbatim}
 
\section{Repeating statements---"while"}
%
\I{while statement}
\I{looping}
You have already seen an implicit looping mechanism---the "select"
block.  Statements inside are executed once for each selected row.
The report writer also contains an explicit loop---the "while" block.
"while" <test>\\
  --- statements executed if <test> is true ---\\
  "end" \opt{"while"}
where
\begin{List}
\item[<test>] is, as with "if", a normal
"where" clause conditions (without the
``where'').  They may contain columns from the current table,
variables, and values.
\end{List}
 
The <test> is evaluated.
If it is `true',
the statements inside the "while" block are executed and the <test>
is re-evaluated. As long as <test> is `true' the statements will
continue to be executed.  As soon as <test> is evaluated `false',
the "while" block is exited and processing continues
with the statement following "end while" statement.
 
\section{Forcing a new page}
%
\I{newpage statement}
Use the
"newpage"
statement to end the current page.
 
\demobreak
This report program, demonstrating the "while" block
and the "newpage" statement,
\label{while-inp}
\begin{verbatim}
*( data entry forms )
report
  select from techs
    print
      1 1 id     i5
      1 10 name  a20
      4 10 'symbol    date    time   resistivity '
      5 1 ' '
    end print
 
    loop = 1
    while loop <= 5
      print
        1 5 loop i2
        2 10 '------  --------  -----  ----------- '
      end print
      loop = loop + 1
    end while
    newpage
  end select
end report
\end{verbatim}
prints the following three-page report.
 
\label{while-rpt}

\begin{verbatim}
     5    John Jones
 
 
          symbol    date    time   resistivity
 
      1
          ------  --------  -----  -----------
      2
          ------  --------  -----  -----------
      3
          ------  --------  -----  -----------
      4
          ------  --------  -----  -----------
      5
          ------  --------  -----  -----------
 
\end{verbatim}
\hrule
\begin{verbatim}
 
    22    Jim Smith
 
 
          symbol    date    time   resistivity
 
      1
          ------  --------  -----  -----------
      2
          ------  --------  -----  -----------
      3
          ------  --------  -----  -----------
      4
          ------  --------  -----  -----------
      5
          ------  --------  -----  -----------
 
\end{verbatim}
\hrule
\begin{verbatim}
 
    35    Joe Jackson
 
 
          symbol    date    time   resistivity
 
      1
          ------  --------  -----  -----------
      2
          ------  --------  -----  -----------
      3
          ------  --------  -----  -----------
      4
          ------  --------  -----  -----------
      5
          ------  --------  -----  -----------
 
\end{verbatim}
 
\section{Procedures}
%
\I{procedures}
A procedure allows you to define and
invoke a group of statements by name.
"procedure" <name> \opt{"using" <table>}\\
 \qquad\vdots\\
 "end" \opt{"procedure"}
defines <name> as a procedure.  Whenever the statement
<name>
is encountered, Rim will execute all the statements of the
procedure.
A procedure cannot have the same name as a variable.
 
If a procedure contains column names from a table,
that table must be identified by the "using" clause.
 
 
\demobreak
This example defines a procedure called "spacer", which prints a blank
line after a paragraphed item.
It is an alternate form of an earlier example and
its output is shown on
page~\pageref{para-demo}.
\demobreak
\begin{verbatim}
report
  procedure spacer   *(define spacer procedure)
    print            *(prints a blank line)
      1 1 ' '
    end print
  end procedure
  select from notes sort by id
    print
      1 5 id     i2
      2 5 symbol a8
      1 15 m_date 'yy/mm/dd'
      2 15 m_time 'hh:mm'
      1 25 notes   a20
    end print
  spacer      *(invoke spacer procedure)
  end select
end report
\end{verbatim}
 
 
 
 
\section{Setting parameters}
%
\I{set statement}
The report writer allows you to alter some of Rim's parameters
during execution of the report program.
Use the "set" statement, which is identical to the
"set" command.  The statements you may use are
"set case" \stk{"ignore"\\"respect"}
"set" \stk{"input"\\"output"} <filename>
"set report" \stk{"width"\\"height"} <count>
which all have the same functionality as the analogous Rim commands.
Remember they take effect when they are executed, not when
they are compiled.
 
 
\section{A `"run"' command}
\I{run@"run"}
\I{macros}
\label{run-mac}
The report writing programs described in previous sections
can be given added flexibility through the use of macros
(chapter~\ref{mac-chapter}).
 
Consider the example in section~\ref{while-inp}
on page~\pageref{while-inp}.
Suppose you wanted to generalize this report to print
an arbitrary number of lines per person, and
to report for only selected persons.
 
One way to do this would be to use
two macros: "\_count" and "\_where".
Then
change line~3 of the report to
" select from techs \_where"
and change line~12 to
" while loop \char` \_count"
 
Define these macros before running the report.
For example,
" macro \_count = 10; macro \_where = ' '"\\
  " input forms"
or
" macro \_count = 7;"\\
  " macro \_where = 'where position = \char`\"Tech 1\char`\"'"\\
  " input forms"
 
\medskip
 
You can make this operation much more friendly through the use
of this "run" macro.%
\footnote{You might want to put this macro definition in your
  login or database initialization files.}
" macro run = +"\\
  "\qquad 'macro arg\_1 = \char`\"' 2 '\char`\";' + "\\
  "\qquad 'macro arg\_2 = \char`\"' 3 '\char`\";' + "\\
  "\qquad 'macro arg\_3 = \char`\"' 4 '\char`\";' + "\\
  "\qquad 'input ' 1"
 
Here is what happens when you type
" run forms 5 'where position like \char`\"Tech*\char`\"' "
\begin{enumerate}
\item The macro "arg\_1" is defined to be `{\tt5}'.
\item The macro "arg\_2" is defined to be
      `{\tt where position like \char`\"Tech*\char`\"}'.
\item The macro "arg\_3" is undefined.
\item The file `forms'\footnote{Some systems may add an
extension to this name.} is input.
 
\end{enumerate}
 
Of course, the report program on file "forms" must be aware of
this "run" invocation.  It should be written as follows.
 
\demobreak
\begin{verbatim}
*( data entry forms )
*( use is:  RUN filename count where_clause )
 
report
  select from techs   arg_2
    print
      1 1 id     i5
      1 10 name  a20
      4 10 'symbol    date    time   resistivity '
      5 1 ' '
    end print
 
    loop = 1
    while loop <=  arg_1
      print
        1 5 loop i2
        2 10 '------  --------  -----  ----------- '
      end print
      loop = loop + 1
    end while
    newpage
  end select
end report
\end{verbatim}
 
to produce the output shown on page~\pageref{while-rpt}.
 
\I{Reports|)}
 
 
 
%
%  <<<<<< Program Interface >>>>>>>>>>>>>
%
 
\chapter{Using Rim with a programming language}
%
\label{pi-chapter}
\I{Fortran|(}
This chapter described the programming language interface
available to Rim users.
Rim is written in \textsc{Fortran-77}, and the descriptions in this
chapter use \textsc{Fortran-77} terminology.  However, you may
invoke Rim functions from any language that can
call \textsc{Fortran-77} function subroutines.
 
This interface is very simple---there is a function subroutine
to execute Rim commands, and another function subroutine
to transfer data.
 
\I{rimcom@"/RIMCOM/"} \I{rmstat@"RMSTAT"}
When Rim detects an error it posts the error number in
the integer variable "RMSTAT", the only element
of the "/RIMCOM/" common block.
 
\begin{verbatim}
   COMMON /RIMCOM/ RMSTAT
   INTEGER RMSTAT
\end{verbatim}
 
If no error occurs, "RMSTAT" will be zero.
Table \ref{rmstat} lists the "RMSTAT" error codes.
Table \ref{rmstat-system} lists "RMSTAT" error codes that
indicate Rim internal errors and should not be encountered.
Normally a status of 0 is good, a status greater than 0 is bad, and
a negative status means there are no more rows to retrieve.
 
\I{rimstp@"/RIMSTP/"} \I{hxflag@"HXFLAG"}
Programs may interrupt Rim's processing by setting
the integer variable "HXFLAG", the only element
of the "/RIMSTP/" common block.
 
\begin{verbatim}
   COMMON /RIMSTP/ HXFLAG
   INTEGER HXFLAG
\end{verbatim}
 
\begin{figure}[p]
\centering
\begin{tabular}{rl}
 -1& no more data available for retrieval                  \\
  0& ok--operation successful                              \\
  1& table not found                                       \\
  2& no database is open                                   \\
  3& column not found                                      \\
  4& syntax error in command                               \\
  5& table already exists                                  \\
  6& HXFLAG interrupt                                      \\
  7& invalid name                                          \\
  8& no authority for operation                            \\
  9& no permission on a table                              \\
 10& files do not contain a Rim database        \\
 12& files incompletely updated (not fatal)       \\
 13& database is attached in read only mode                \\
 14& database is being updated                             \\
 15& tuple too long                                        \\
 16& database has not been opened                          \\
 20& undefined table                                    \\
 30& undefined column                                   \\
 40& where clause too complex                           \\
 42& unrecognized comparison operator                      \\
 43& `like' only available for text columns           \\
 45& unrecognized logical operator                          \\
 46& compared columns must be the same type/length      \\
 47& lists are valid only for eq and ne                    \\
 50& "select" not called                                     \\
 60& "get" not called                                      \\
 70& multiple table index out of range                \\
 80& variable length columns may not be sorted          \\
 81& the number of sorted columns is too large          \\
 89& sort system error                                     \\
 90& unauthorized table access                          \\
 91& table already exists                               \\
 92& bad column type                                    \\
 93& bad column length                                  \\
 94& too many or two few columns                        \\
 95& row too big to define                                 \\
100& illegal variable length row definition (load/put)     \\
\end{tabular}
\caption{{\tt RMSTAT} error codes}
\label{rmstat}
\end{figure}
 
\begin{figure}[t]
\centering
\begin{tabular}{rl}
1001& buffer size problem -- BLKCHG,BLKDEF                 \\
1002& undefined block -- BLKLOC                            \\
1003& cannot find a larger b-tree value -- BTADD,PUTDAT    \\
1004& cannot find b-tree block -- BTPUT                    \\
21xx& random file error xx on file1                        \\
22xx& random file error xx on file2                        \\
23xx& random file error xx on file3  \\
3000& sort error - no buffer available \\
31xx& sort error - xx on file open
\end{tabular}
\caption{{\tt RMSTAT} error codes indicating Rim internal errors.}
\label{rmstat-system}
\end{figure}
 
Users of the program interface may independently access several
tables simultaneously.  An indexing integer specifies which
table is to be accessed.
The index is ignored for
those commands which do not access
a specific table (e.g. "open").
The index has a range of 1--10 (ZPIMAX).
 
 
\section{Executing Rim commands}
%
\I{rim@"RIM"}
The logical function
"RIM(" <index>, <command> ")"
where
<index> is the integer table index,\\
       <command> is a string containing a Rim command.
returns a value of true if the command is executed successfully,
and false if there has been an error.  In the latter case
"RMSTAT" will contain the error number.
 
For example, to open the "Sample" database and enter the
user password of "metals"\footnote{Our Sample database
as described in this reference had no password.}
you can include
 
\begin{verbatim}
  IF (.NOT.RIM(0,'open sample')) GOTO  <error label>
  IF (.NOT.RIM(0,'user metals')) <ditto>
\end{verbatim}
 
in your \textsc{Fortran} program.
 
\section{Transferring data}
%
\I{rimdm@"RIMDM"}
The logical function
"RIMDM(" <index>, <command>, <tuple> ")"
where
<index> is the integer table index,\\
       <command> is a string containing a Rim data-movement command, and\\
       <tuple> is the integer array containing the data to be transferred.
returns a value of `true' if the command is executed successfully,
and `false' if there has been an error.  In the latter case
"RMSTAT" will contain the error number.
You must have executed a Rim `table-selection' command
(presently only "select" and "load")
before issuing a data-movement command.
 
The data-movement commands are:
 
\medskip
\begin{tabular}{lp{12pc}ll}
Command&
Description&
\multicolumn{2}{l}{Prerequisite}\\
\hline
"get"& Retrieves the next row from the table.& "select"&("RIM")\\
"put"& Replaces the current row into the table.& "get"&("RIMDM")\\
"del"& Deletes the current row from the table.& "get"&("RIMDM")\\
"load"& Loads a new row at the end of the current table.&
                "load"&("RIM")
\end{tabular}
 
\medskip
 
\subsubsection{Format of the tuples}
%
\I{tuple}
The word `tuple' is used to mean an array containing a row of table data.
Most columns are located sequentially in the tuple and
occupy a fixed number of words depending on the column's type.
Integers occupy one word per number.  Double precision numbers
occupy two words per number.  Text is packed {\small CPW}
characters per word, where {\small CPW} is a machine dependent
parameter.  On 32-bit machines $\mbox{\small CPW}=4$. We will
assume the value of 4 in this discussion.
A tuple containing an integer (value <i>), a 10-character text item
(value <text>),
a double (value <r>),
and a date (value <d>) occupies 7~words and looks like this:
 
\smallskip
{\centering\small\tabcolsep2pt
 \def\strut{\rule{0pt}{2pt}} \def\Strut{\rule{0pt}{12pt}}
 \def\fl{\multicolumn{1}{|l|}}
\begin{tabular}{|c|c|c|c|c|c|c|}
 \fl{1}&\fl{2}&\fl{3}&\fl{4}&\fl{5}&\fl{6}&\fl{7}\\
 \hline
   \rule{38pt}{0pt}&
   \rule{38pt}{0pt}&
   \rule{38pt}{0pt}&
   \rule{38pt}{0pt}&
   \rule{38pt}{0pt}&
   \rule{38pt}{0pt}&
   \rule{38pt}{0pt}\\
 \Strut att 1&
 \multicolumn{3}{|c|}{att 2}&
 \multicolumn{2}{|c|}{att 3}&
 att 4\\
 \Strut [<i>]&
 \multicolumn{3}{|c|}{[<text>]}&
 \multicolumn{2}{|c|}{[<r>]}&
 [<d>]\\
 \Strut (1 word)&
 \multicolumn{3}{|c|}{(3 words)}&
 \multicolumn{2}{|c|}{(2 words)}&
 (1 word)\\
 &&&&&&\\
 \hline
\end{tabular}
 \par}
 
\medskip
Variable length columns are the exception to this rule.
They have a one-word pointer at their position in the tuple.
This pointer contains the offset (from the start of the tuple)
to the actual location of the variable column's data.
At that offset a two word header contains length information.
Following this is the actual value.
The same tuple, if the character string is variable length,
occupies 10~words and
looks like this:
 
\smallskip
{\centering\small\tabcolsep2pt
 \def\strut{\rule{0pt}{2pt}} \def\Strut{\rule{0pt}{12pt}}
 \def\fl{\multicolumn{1}{|l|}}
\begin{tabular}{|c|c|c|c|c|c|c|c|c|c|}
 \fl{1}&\fl{2}&\fl{3}&\fl{4}&\fl{5}&\fl{6}&\fl{7}&\fl{8}&\fl{9}&\fl{10}\\
 \hline
   \rule{25pt}{0pt}&
   \rule{25pt}{0pt}&
   \rule{25pt}{0pt}&
   \rule{25pt}{0pt}&
   \rule{25pt}{0pt}&
   \rule{25pt}{0pt}&
   \rule{25pt}{0pt}&
   \rule{25pt}{0pt}&
   \rule{25pt}{0pt}&
   \rule{25pt}{0pt}\\
 \Strut att 1&
 att 2&
 \multicolumn{2}{|c|}{att 3}&
 att 4&
  W1 &
  W2 &
  \multicolumn{3}{|c|}{att 3 (value)}\\
 \Strut [int]&
 [6]&
 \multicolumn{2}{|c|}{[double]}&
 [date]&
   [10]&
   [0]&
   \multicolumn{3}{|c|}{[text]}\\
 \Strut (1 word)&
 (1 word)&
 \multicolumn{2}{|c|}{(2 words)}&
 (1 word)&
  (1 word)&
  (1 word)&
  \multicolumn{3}{|c|}{(3 words)}\\
  &&&&&&&&&\\
 \hline
\end{tabular}
\par}
 
\medskip
The variable length header (words W1 and W2 in the example) contains
 
{\centering
\begin{tabular}{lll}
Data type& W1& W2\\
\noalign{\medskip}
text& \# chars& 0\\
vector&\# cols& 0\\
matrix& \# rows& \#cols
\end{tabular}}
 
 
\medskip
 
\subsubsection{Text columns}
%
\I{text}
Data in text columns are represented by integer values.
Before you can use these in your program you must convert them
to characters in character variables.
The subroutine
\I{strasc@"STRASC"}
"CALL STRASC(" <string>, <tuple> "("<pos>")", <len> ")"
converts a text column at <tuple>"("pos")" of length <len> characters
into a character string in <string>, type CHARACTER.
 
To move characters back into the tuple use the inverse subroutine
\I{asctxt@"ASCTXT"}
"CALL ASCTXT(" <tuple>"("<pos>")," <len>, <string> ")"
which moves the characters in <string> into the tuple at <pos>,
length <len>.
 
 
\subsubsection{Date columns}
%
\I{date}
Dates are represented by integer Julian values,  according to
the algorithm of Tantzen.\footnote{Collected Algorithms of
the ACM, 199, R. Tantzen.}
Use the subroutine
\I{datjul@"DATJUL"}
"CALL DATJUL(" <day>, <month>, <year>, <Julian> ")"
to convert a Julian integer into day, month, and year integers.
 
Use the inverse subroutine
\I{juldat@"JULDAT"}
"CALL JULDAT( <day>, <month>, <year>, <Julian> )"
to convert the day, month, and year into a Julian integer.
 
\subsubsection{Time columns}
%
\I{time}
Times are also represented by integer values,  according to
the formula
$$\hbox{time integer} =
      \hbox{hours}*3600 + \hbox{minutes}*60 + \hbox{seconds}.$$
There are no subroutines provided to facilitate conversion.
 
\subsubsection{Missing values}
%
\I{missing value}
A ``missing value'' is indicated by the integer value
ZIMISS (2147483647) and a ``not applicable'' value
is indicated by the integer value ZINAPP (2147483646).
\I{MV}\I{NA}
 
\subsection{Retrieving the rows of a table}
%
\I{select@"select"}
Open the database and set the password, if one is required, and then
issue the select command
"select from" <table> \\
    \qquad \opt{"where" <where clause>} \\
    \qquad \opt{"sort" \opt{"by"} <sort clause>}
via the "RIM" function.
No column list is allowed as Rim always transfers complete rows.
 
Now you can "get" rows from this table, modify them
and "put" them back, or "del"ete them.
A prototype retrieval loop is illustrated in Figure \ref{pi-ret}.
Note that the character variables must be unpacked from the integer
tuple array with "STRASC" and must be replaced with "ASCTXT".
 
 
\subsection{Loading data into a table}
%
\I{load@"load"}
Open the database and set the password, if one is required, and then
issue the load command
"load" <table>
via the "RIM" function.
 
Now you can "load" rows into this table with the
"RIMDM" function.
 
\begin{figure}
  %\let\bf\tt
$\vdots$\\
    "IF (.NOT.RIM(1,'open example')) GOTO "<error>\\
    $\vdots$\\
    "IF (.NOT.RIM(1," \\
       \llap{"1"}
       "'select from conductors sort by symbol')) GOTO "<error>\\
    \\
    \llap{<label>\ \ }"IF (RIMDM(1,'get',TUPLE)) THEN"  \\
    "~~~"$\vdots$\\
    "~~~IF (<modifying>) THEN"\\
    "~~~~~~IF (.NOT.RIMDM(1,'put',TUPLE)) GOTO "<error>\\
    "~~~ELSE IF (<deleting>) THEN "\\
    "~~~~~~IF (.NOT.RIMDM(1,'del',TUPLE)) GOTO "<error>\\
    "~~~ENDIF"\\
    "~~~GOTO "<label>\\
    "ELSE" \\
    "~~~IF (RMSTAT.NE.-1) GOTO "<error>\\
    "ENDIF"
    $\vdots$
\caption{Sample data retrieval with update}
\label{pi-ret}
\end{figure}
 
\I{Fortran|)}
 
 
% <<<<<< Appendix A >>>>>>>>
 
\appendix
 
\chapter{Using Rim at UCS}
%
\label{UCS-systems}
\I{UCS}
This chapter tells you how to use Rim on the various UCS
computers.
 
\section{Using Rim with UNIX}
%
\I{UNIX}
\I{Running Rim}
 
\subsection{Running Rim}
Start Rim with the command
\% "rim" 
\I{initialization}
If the initialization file ".rimrc" exists in your home directory, it will
be input when Rim starts.
 
You can also run Rim in a batch mode by specifying an input file.
To do this start Rim with the command
\% "rim" <input\_file>
Rim will process the <input\_file> and will not communicate
with your terminal.
 
\subsection{Identifying files}
%
\I{filename}
Filenames, and the database name, may include directory information.
For example,
"open 'user/elsewhere/hisdb'"
opens the database "hisdb" on the directory "user/elsewhere".
The name must be enclosed in quotes only if it
contains special characters.
 
UNIX Rim does not assume any extensions for either input or
output files.
Database files have the extensions ".rimdb1", ".rimdb2", and ".rimdb3".
 
 
\subsection{Using the \textsc{Fortran} interface}
%
\I{Fortran}
To load \textsc{Fortran} programs with the Rim interface include the
"rimlib.a" library with the "-lrim" option
in your "ld" or "f77" command.
 
 
\subsection{Executing shell commands within Rim}
%
\I{Executing system commands}
\I{system@"system"}
UNIX Rim allows you to execute shell
commands by
"system" <command>
where <command> will normally be a quoted string.
Rim closes your database prior to executing
the system command so your data should be OK even if you
cannot return.
 
 
\section{Using Rim with VAX/VMS}
%
\I{VAX/VMS}
\I{Running Rim}
 
\subsection{Running Rim}
Start Rim with the command
\$ "rim" 
\I{initialization}
If the initialization file "login.rim" exists in your login directory, it will
be input when Rim starts.
 
You can also run Rim in a batch mode by specifying an input file.
To do this start Rim with the command
\$ "rim" <input\_file>
Rim will process the <input\_file> and will not communicate
with your terminal.
 
\subsection{Identifying files}
%
\I{filename}
Filenames, and the database name, may include directory information.
For example,
"open '[zz99.elsewhere]hisdb'"
opens the database "hisdb" on the directory "[zz99.elsewhere]".
The name must be enclosed in quotes only if it
contains special characters.
 
VMS Rim assumes an extension of ".rim" for input files and
".lis" for output files.
Database files have the extensions ".rimdb1", ".rimdb2", and ".rimdb3".
 
 
\subsection{Using the \textsc{Fortran} interface}
%
\I{Fortran}
To load \textsc{Fortran} programs with the Rim interface, include the
library
"\$\$rim:rimlib/lib"
in your link command.
 
\section{Using Rim with IBM VM/CMS}
%
\I{VM/CMS}
\I{Running Rim}
 
\subsection{Running Rim}
Before running Rim you must run "setup" to access the appropriate disks.
"setup rim"
 
Start Rim with the command
"rim" 
\I{initialization}
If the initialization file "profile rim" exists, it will
be input when Rim starts.
 
You can also run Rim in a batch mode by specifying an input file.
To do this start Rim with the command
"rim" <input\_file>
Rim will process the <input\_file> and will not communicate
with your terminal.
 
 
Then start Rim with the command
"rim" <input\_file>
and Rim will read the <input\_file> if it is given.
<input\_file> may contain a type and mode.
 
You use "setup" only once per session.  It may be included in your
"profile exec" file.
 
\subsection{Identifying files}
%
\I{filename}
Filenames may include types and modes and may use spaces or periods
as delimiters.
"'file type m'" and "file.type.m"
are equivalent.
 
The database specification may include a mode letter.
For example,
"open hisdb c"
opens the database "hisdb" on the disk "c".
 
CMS Rim assumes a file type  of "rim" for input files and
"rimlist" for output files.
Database files have the types "rimdb1", "rimdb2", and "rimdb3".
 
\subsection{Using the \textsc{Fortran} interface}
%
\I{Fortran}
To load \textsc{Fortran} programs with the Rim interface, include the
libraries RIMLIB and UTILITY
in your GLOBAL TXTLIB command.
RIMLIB and this sample loader EXEC (RIMLOAD EXEC) may be
found on the Rim disk.
 
\begin{verbatim}
/* load program using rimlib */
arg fname "(" options
"GLOBAL TXTLIB RIMLIB VLNKMLIB VFORTLIB CMSLIB UTILITY"
"LOAD" fname "(" options
exit
\end{verbatim}
 
\subsection{Executing CMS commands within Rim}
%
\I{Executing system commands}
\I{system@"system"}
CMS Rim allows you to execute \underline{selected} CMS or
CP commands by
"system" <command>
where <command> will normally be a quoted string.
You can run XEDIT and most EXECs, but you cannot run other
programs.  Rim closes your database prior to executing
the system command so your data should be OK even if you
cannot return.
 
\medskip
 
Since Rim will read from the program stack you can write
`Rim EXECs' which build and execute Rim commands.
One example of this capability is an interactive
data loader for the "notes" table.
%
\begin{figure}[htp]
\narrower
\begin{verbatim}
/* Rim exec to load notes entries */
say "Enter the symbol and your id"
parse pull symbol id
if id = '' then exit
D = date(O)
T = substr(time(),1,5)
say "Enter notes"
parse pull notes
queue "load notes"
queue symbol id "'"D"' '"T"' +"
queue "'"notes"'"
queue "end"
queue "system 'exec l_notes'"
exit
\end{verbatim}
\caption{Sample VM/CMS Rim EXEC}
{This EXEC, named ``L\_NOTES EXEC'' will
interactively load the notes table of the Sample relation.\par}
\label{CMS-exec}
\end{figure}
%
Figure \ref{CMS-exec} shows the EXEC which reads the data,    formats
a load command, and returns it to Rim for execution. It also
stacks a command to re-execute itself unless it has been given
a null response.
 
\chapter{Distribution and License}
%
\I{Distribution}
\section{Distribution}
 
Rim is free; this means that everyone is free to use it and
free to redistribute it on a free basis.  Rim is not in the public
domain; it is copyrighted and there are restrictions on its
distribution---restrictions similar to those of GNU software.
 
The easiest way to get a copy of Rim is from someone else who has it.
You need not need permission.
If you cannot get a copy this way, you can order one from
University Computing Services.
Though Rim itself is free, our distribution
service is not.
For further information, contact
 
\begin{verse}
University of Washington\\
University Computing Services, HG--45\\
3737 Brooklyn Ave NE\\
Seattle, WA  98105\\
USA
\end{verse}
 
\section{Rim General Public License}
\I{License}
Out intention is
to give everyone the right to share Rim.
To make
sure that you get the rights we want you to have, we need to make
restrictions that forbid anyone to deny you these rights or to ask you
to surrender the rights.  Hence this license agreement.
 
For our own protection, we must make certain that everyone
finds out that there is no warranty for Rim.
 
\subsection{Copying Policies}
 
\begin{enumerate}
\item
You may copy and distribute verbatim copies of Rim source code as you
receive it, in any medium, provided that you conspicuously and
appropriately publish on each file a valid copyright notice ``Copyright
\copyright 1988 University of Washington'' (or with whatever year
is appropriate); keep intact the notices on all files that
refer to this License Agreement and to the absence of any warranty; and
give any other recipients of Rim a copy of this License
Agreement along with the program.  You may charge a distribution fee
for the physical act of transferring a copy.
 
\item
You may modify your copy or copies of Rim source code or
any portion of it, and copy and distribute such modifications under
the terms of Paragraph 1 above, provided that you also do the following:
 
\begin{itemize}
\item
cause the whole of any work that you distribute or publish, that
in whole or in part contains or is a derivative of Rim or any
part thereof, to be licensed at no charge to all third parties on
terms identical to those contained in this License Agreement
(except that you may choose to grant more extensive warranty
protection to some or all third parties, at your option).
 
\item
You may charge a distribution fee for the physical act of
transferring a copy, and you may at your option offer warranty
protection in exchange for a fee.
\end{itemize}
 
Mere aggregation of another unrelated program with this program (or its
derivative) on a volume of a storage or distribution medium does not bring
the other program under the scope of these terms.
 
\item
You may copy and distribute Rim (or a portion or derivative of it,
under Paragraph 2) in object code or executable form under the terms
of Paragraphs 1 and 2 above provided that you also do one of the
following:
 
\begin{itemize}
\item
accompany it with the complete corresponding machine-readable
source code, which must be distributed under the terms of
Paragraphs 1 and 2 above; or,
 
\item
accompany it with a written offer, valid for at least three
years, to give any third party free (except for a nominal
shipping charge) a complete machine-readable copy of the
corresponding source code, to be distributed under the terms of
Paragraphs 1 and 2 above; or,
 
\item
accompany it with the information you received as to where the
corresponding source code may be obtained.  (This alternative is
allowed only for noncommercial distribution and only if you
received the program in object code or executable form alone.)
\end{itemize}
 
For an executable file, complete source code means all the source code
for all modules it contains; but, as a special exception, it need not
include source code for modules which are standard libraries that
accompany the operating system on which the executable file runs.
 
\item
You may not copy, sublicense, distribute or transfer Rim except
as expressly provided under this License Agreement.  Any attempt
otherwise to copy, sublicense, distribute or transfer Rim is
void and your rights to use Rim under this License agreement
shall be automatically terminated.  However, parties who have received
computer software programs from you with this License Agreement will
not have their licenses terminated so long as such parties remain in
full compliance.
 
\end{enumerate}
 
 
%
%   Index
%
%\cleardoublepage
%\printindex
 
\end{document}

 
